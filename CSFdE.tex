\documentclass[dvipsnames]{article}
\usepackage[a4paper, landscape]{geometry}
\usepackage{url}
\usepackage{multicol} 
\usepackage{amsmath}
\usepackage{esint}
\usepackage{amsfonts}
\usepackage{tikz}
\usetikzlibrary{decorations.pathmorphing}
\usepackage{caption}
\usepackage{amsmath,amssymb}
\usepackage{bbding}
\usepackage{pifont}
\usepackage{wasysym}
\usepackage{colortbl}
\usepackage{xcolor, soul} %colors and \hl
\usepackage{mathtools} 
\usepackage{amsmath,amssymb}
\usepackage[shortlabels]{enumitem}
\usepackage{fontawesome}
\usepackage[american,european, siunitx]{circuitikz} %https://grex99.gitlab.io/circuitgui/
\usepackage{graphicx}
\usepackage{microtype}

\makeatletter

\newcommand*\bigcdot{\mathpalette\bigcdot@{.5}}
\newcommand*\bigcdot@[2]{\mathbin{\vcenter{\hbox{\scalebox{#2}{$\m@th#1\bullet$}}}}}
\makeatother

\definecolor{CustomYellow}{RGB}{233,215,0}

\title{Cheat Sheet}
\usepackage[brazilian]{babel}
\usepackage[utf8]{inputenc}

\advance\topmargin-.8in
\advance\textheight3in
\advance\textwidth3in
\advance\oddsidemargin-1.5in
\advance\evensidemargin-1.5in
\parindent0pt
\parskip2pt
%------------ COMMANDS ---------------
\newcommand{\hr}{\centerline{\rule{3.5in}{1pt}}}
%\colorbox[HTML]{e4e4e4}{\makebox[\textwidth-2\fboxsep][l]{texto}


%------------ CONDITIONAL FORMATTING ---------------
\newcommand\conform{}
%\newcommand\conform{S}
\newcommand\dummytext[1]{\if \conform S {#1} \else \vspace{5cm} \hspace{5cm} \fi}

%---------------------------------------------------
\begin{document}

%\begin{center}{\huge{\textbf{Formulario esame elettrotecnica}}}\\
%\end{center}
\begin{multicols*}{3}

    \tikzstyle{mybox} = [draw=CustomYellow, fill=white, very thick,
    rectangle, rounded corners, inner sep=10pt, inner ysep=10pt]
    \tikzstyle{fancytitle} =[fill=CustomYellow, text=white, font=\bfseries]

    \tikzset{
        miaFreccia/.style={
                ->,
                >=latex,
                line width=1pt,
                shorten <=5mm,
                shorten >=5mm
            }
    }

    %SCALA TRANSITORI

    \resizebox{\columnwidth}{0.9\textheight}{%------------ Box 26 ---------------
        \label{Box 26}
        \begin{tikzpicture}
            \node [mybox] (box){%
                \begin{minipage}{0.3\textwidth}

                    \begin{enumerate}
                        \item Per \colorbox{yellow}{$t \to 0^-$},
                              \begin{enumerate}
                                  \item \textbf{calcolare variabile di stato prima dell’inizio del transitorio}
                                  \item In questa fase il \textbf{\color{orange}condensatore\color{black}}/\textbf{\color{Fuchsia}induttore\color{black}                                   \color{black}} si comporta come \textbf{\color{orange}circuito aperto\color{black}}/\textbf{\color{Fuchsia}cortocircuito\color{black}                                   \color{black}}
                                  \item Sfrutterò nella fase 2 la continuitá della variabile di stato
                              \end{enumerate}

                        \item Per \colorbox{Dandelion}{\(t \to 0^+\) (per var. \textbf{NON} di stato es. $v_x, i_x$) },
                              \begin{enumerate}
                                  \item (Eventuale chiusura interruttore)
                                  \item \textbf{Sfrutto continuitá variabile di stato}: \\ \(\color{orange}v_C(t_0^-)=v_C(t_0^+)\)\color{black} / \color{Fuchsia}\(i_L(t_0^-)= i_L(t_0^+)\)\color{black}
                                  \item \textbf{Sostituisco al transitorio} GENERATORE IDEALE DI \color{orange}\textbf{TENSIONE}\color{black} / \color{Fuchsia}\textbf{CORRENTE }\color{black} con \ul{\textbf{valore pari alla variabile di stato appena calcolata}} $$E=V_C(t \to 0^-) \quad I=I_L(t \to 0^-)$$
                              \end{enumerate}

                        \item Per \colorbox{BurntOrange}{\(t \to \infty\)} / \colorbox{BurntOrange}{\(t>0\)} :
                              \begin{enumerate}
                                  \item \textbf{Soluzione di tipo esponenziale}
                                        \begin{enumerate}
                                            \item Formule variabili di stato:
                                                  \begin{align*}
                                                      V_C(t) & = V_{C_{\infty}}+\left[V_{C}(0)-V_{C_\infty}\right] e^{-\frac{t}{\tau}} \\
                                                      I_L(t) & = I_{L_{\infty}}+\left[I_{L}(0)-I_{L_\infty}\right] e^{-\frac{t}{\tau}}
                                                  \end{align*}

                                            \item Formule per le grandezze \textbf{non di stato}:
                                                  \begin{align*}
                                                      I_C(t) & = I_{C_\infty}+[I_C(\color{red}0^+\color{black})-I_{C_\infty}]e^{\frac{-t}{\tau}} \\
                                                      V_L(t) & = V_{L_\infty}+[V_L(\color{red}0^+\color{black})-V_{L_\infty}]e^{\frac{-t}{\tau}}
                                                  \end{align*}

                                            \item Qui, siamo \textbf{\color{red}ancora a regime\color{black}}: il \textbf{\color{orange}condensatore\color{black}}/\textbf{\color{Fuchsia}induttore\color{black}                                   \color{black}} si comporta come \textbf{\color{orange}circuito aperto\color{black}}/\textbf{\color{Fuchsia}cortocircuito\color{black}                                   \color{black}}
                                            \item Cerco la variabile di stato per \(t \to \infty\)
                                            \item Cerco \(\tau\):
                                                  \begin{enumerate}
                                                      \item Mi serve \(R_{\text{eq}}\) ai morsetti di dove c’é transitorio
                                                      \item \textbf{\color{brown}Spengo generatori non pilotati\color{black}}
                                                      \item uso \color{teal} \textbf{generatore sonda (c.g.)} - cerco corrente che passa sul ramo della sonda in funzione di \(V_S\):  \(?\rightarrow I_S(V_S)\)

                                                            \[
                                                                R_{\text{eq}}=\frac{V_S}{I_S(V_S)}
                                                            \]
                                                            \color{black}
                                                      \item Calcolo  \(\tau\):

                                                            \[
                                                                \tau = C \cdot R_{\text{eq}} = \frac{L}{R_{\text{eq}}}
                                                            \]
                                                  \end{enumerate}
                                        \end{enumerate}


                              \end{enumerate}
                    \end{enumerate}
                \end{minipage}
            };
            %------------ Box 26 Header ---------------------
            \node[fancytitle, right=10pt] at (box.north west) {\color{white}Procedimento transitori:};
        \end{tikzpicture}}


    %------------ Box 27 ---------------
    \label{Box 27}
    \begin{tikzpicture}
        \node [mybox] (box){%
            \begin{minipage}{0.3\textwidth}
                \begin{enumerate}
                    \item Traccio asintoto
                    \item Sfrutto \textbf{proprietá dell’esponenziale}: tangente al grafico in \(t=0\) interseca il valore asintotico dopo \(\Delta t = \tau\)
                    \item Dopo \(t=5\tau\) la funzione assume valore asintotico
                \end{enumerate}

            \end{minipage}
        };
        %------------ Box 27 Header ---------------------
        \node[fancytitle, right=10pt] at (box.north west) {\color{white}Grafico};
    \end{tikzpicture}
    %------------ Box 3 ---------------
    \label{Box 3}
    \begin{tikzpicture}
        \node [mybox] (box){%
            \begin{minipage}{0.3\textwidth}


                \dummytext{Text 3}


            \end{minipage}
        };
        %------------ Box 3 Header ---------------------
        \node[fancytitle, right=10pt] at (box.north west) {\color{white}Box 3};
    \end{tikzpicture}

    %%%%%%%%%%%%%%%%%%%%%%%%%%%%%%%%%%%%%%%%%%%%%%%%%%%%%%%%%%%%%%%%%%%%




    %------------ Box 4 ---------------
    \label{Box 4}
    \begin{tikzpicture}
        \node [mybox] (box){%
            \begin{minipage}{0.3\textwidth}


                \dummytext{Text 4}


            \end{minipage}
        };
        %------------ Box 4 Header ---------------------
        \node[fancytitle, right=10pt] at (box.north west) {\color{white}Box 4};
    \end{tikzpicture}

    %%%%%%%%%%%%%%%%%%%%%%%%%%%%%%%%%%%%%%%%%%%%%%%%%%%%%%%%%%%%%%%%%%%%





    %------------ Box 5 ---------------------
    \label{Box 5}
    \begin{tikzpicture}
        \node [mybox] (box){%
            \begin{minipage}{0.3\textwidth}


                \dummytext{Text 5}


            \end{minipage}
        };
        %------------ Box 5 Header ---------------------
        \node[fancytitle, right=10pt] at (box.north west) {\color{white}Box 5};
    \end{tikzpicture}

    %%%%%%%%%%%%%%%%%%%%%%%%%%%%%%%%%%%%%%%%%%%%%%%%%%%%%%%%%%%%%%%%%%%%




    %------------ Box 6 ---------------
    \label{Box 6}
    \begin{tikzpicture}
        \node [mybox] (box){%
            \begin{minipage}{0.3\textwidth}


                \dummytext{Text 6}


            \end{minipage}
        };
        %------------ Box 6 Header ---------------------
        \node[fancytitle, right=10pt] at (box.north west) {\color{white}Box 6};
    \end{tikzpicture}

    %%%%%%%%%%%%%%%%%%%%%%%%%%%%%%%%%%%%%%%%%%%%%%%%%%%%%%%%%%%%%%%%%%%%





    %------------ Box 7 ---------------
    \label{Box 7}
    \begin{tikzpicture}
        \node [mybox] (box){%
            \begin{minipage}{0.3\textwidth}


                \dummytext{Text 7}


            \end{minipage}
        };
        %------------ Box 7 Header ---------------------
        \node[fancytitle, right=10pt] at (box.north west) {\color{white}Box 7};
    \end{tikzpicture}

    %%%%%%%%%%%%%%%%%%%%%%%%%%%%%%%%%%%%%%%%%%%%%%%%%%%%%%%%%%%%%%%%%%%%



    %------------ Box 8 ---------------
    \label{Box 8}
    \begin{tikzpicture}
        \node [mybox] (box){%
            \begin{minipage}{0.3\textwidth}


                \dummytext{Text 8}


            \end{minipage}
        };
        %------------ Box 8 Header ---------------------
        \node[fancytitle, right=10pt] at (box.north west) {\color{white}Box 8};
    \end{tikzpicture}

    %%%%%%%%%%%%%%%%%%%%%%%%%%%%%%%%%%%%%%%%%%%%%%%%%%%%%%%%%%%%%%%%%%%%






    %%%%%%%%%%%%%%%%%%%%%%%%%%%%%%%%%%%%%%%%%%%%%%%%%%%%%%%%%%%%%%%%%%%%

    %------------ Formazione del Canale nei MOSFET ---------------
    \label{Formazione del Canale nei MOSFET}
    \begin{tikzpicture}
        \node [mybox] (box){%
            \begin{minipage}{0.3\textwidth}



                \begin{enumerate}
                    \item \textbf{Zona OFF (o Cutoff):}
                          \begin{enumerate}
                              \item Non c'è formazione del canale.
                              \item Il dispositivo è spento e non permette il flusso di corrente tra drain e source.
                          \end{enumerate}

                    \item \textbf{Zona Ohmica (o Triodo):}
                          \begin{enumerate}
                              \item Si forma un canale.
                              \item Quando il gate è abbastanza polarizzato (cioè \( V_{GS} > V_{Tn} \) per nMOS o \( V_{GS} < V_{Tp} \) per pMOS), si forma un canale conduttivo tra il drain e il source.
                              \item Il dispositivo si comporta come un \textbf{resistore il cui valore varia in base alla tensione} \( V_{GS} \).
                          \end{enumerate}

                    \item \textbf{Zona di Saturazione (o Pinch-off):}
                          \begin{enumerate}
                              \item Si forma un canale.
                              \item Il canale diventa "strozzato" o "pinched-off" vicino al drain (per il nMOS) o vicino al source (per il pMOS).
                              \item Anche se la tensione \( V_{DS} \) aumenta ulteriormente, la corrente \( I_D \) rimane costante.
                              \item Questo comportamento è \textbf{analogo a quello di un generatore di corrente.}
                          \end{enumerate}
                \end{enumerate}



            \end{minipage}
        };
        %------------ Formazione del Canale nei MOSFET Header ---------------------
        \node[fancytitle, right=10pt] at (box.north west) {\color{white}Formazione del Canale nei MOSFET};
    \end{tikzpicture}

    %%%%%%%%%%%%%%%%%%%%%%%%%%%%%%%%%%%%%%%%%%%%%%%%%%%%%%%%%%%%%%%%%%%%




    %------------ Overdrive ---------------------
    \label{Overdrive}
    \begin{tikzpicture}
        \node [mybox] (box){%
            \begin{minipage}{0.3\textwidth}

                \noindent
                \textbf{\color{red}Overdrive}: \textbf{quanto fortemente é acceso il transistor.} Differenza fra \textbf{\color{red}tensione G/S e soglia}.
                $$V_{OV} = V_{GS}-V_{T}$$
                dove $V_T = V_{Tn}, V_T = V_{Tp}$ \\ a seconda che si tratti di \textbf{nMOS} o \textbf{pMOS}.    \\

                Ogni qualvolta in cui un transistor si trova in saturazione, la corrente che scorre da D a S sará:
                $$I_D = k (V_{GS} - V_T)^2=k \cdot {V_{OV}}^2$$
            \end{minipage}
        };
        %------------ Overdrive Header ---------------------
        \node[fancytitle, right=10pt] at (box.north west) {\color{white}Overdrive};
    \end{tikzpicture}
    \resizebox{\columnwidth}{0.9\textheight}{
        %%%%%%%%%%%%%%%%%%%%%%%%%%%%%%%%%%%%%%%%%%%%%%%%%%%%%%%%%%%%%%%%%%%%

        %------------ Box 12 ---------------------
        \label{Box 12}
        \begin{tikzpicture}
            \node [mybox] (box){%
                \begin{minipage}{0.3\textwidth}
                    \noindent \textbf{Parametri di Transconduttanza:}
                    \begin{equation*}
                        \begin{aligned}
                             & K_n = \frac{1}{2} \mu_n C_{OX} \frac{W}{L} \quad \text{e} \quad K_p = -\frac{1}{2} \mu_p C_{OX} \frac{W}{L} \\
                             & \begin{aligned}
                                   \mu_n  & : \text{Mobilità degli elettroni nel nMOS}           \\
                                   \mu_p  & : \text{Mobilità delle lacune nel pMOS}              \\
                                   C_{OX} & : \text{Capacità per unità di area dell'ossido gate} \\
                                   W      & : \text{Larghezza del canale}                        \\
                                   L      & : \text{Lunghezza del canale}
                               \end{aligned}
                        \end{aligned}
                    \end{equation*}


                    \noindent \textbf{nMOS:}
                    \begin{enumerate}
                        \item \textbf{Spegnimento (Cutoff):}
                              \begin{equation*}
                                  \begin{aligned}
                                       & V_{GS} < V_{Tn}                           \\
                                       & I_D = 0                                   \\
                                       & \text{Circuito aperto tra drain e source}
                                  \end{aligned}
                              \end{equation*}
                        \item \textbf{Zona Ohmica (Triodo):}
                              \begin{equation*}
                                  \begin{aligned}
                                       & V_{GS} > V_{Tn}, \quad V_{DS} < V_{OV}                        \\
                                       & I_D = k_P \left[ 2 (V_{GS}-V_{Tn})V_{DS} - {V_{DS}}^2 \right] \\
                                       & \text{Resistore controllato da tensione}
                                  \end{aligned}
                              \end{equation*}
                        \item \textbf{Zona di Saturazione (Pinch-off):}
                              \begin{equation*}
                                  \begin{aligned}
                                       & V_{GS} > V_{Tn}, \quad V_{DS} > V_{OV}, \quad V_{GD} < V_{Tn} \\
                                       & \text{Generatore di corrente controllato}
                                  \end{aligned}
                              \end{equation*}
                    \end{enumerate}

                    \noindent \textbf{pMOS:} $\color{Purple}I_D <0, V_{DS}<0$
                    \begin{enumerate}
                        \item \textbf{Spegnimento (Cutoff):}
                              \begin{equation*}
                                  \begin{aligned}
                                       & V_{GS} > |V_{Tp}|                         \\
                                       & I_D = 0                                   \\
                                       & \text{Circuito aperto tra drain e source}
                                  \end{aligned}
                              \end{equation*}
                        \item \textbf{Zona Ohmica (Triodo):}
                              \begin{equation*}
                                  \begin{aligned}
                                       & V_{GS} < V_{Tp}, \quad V_{SD} < V_{OV}                        \\
                                       & I_D = k_P \left[ 2 (V_{GS}-V_{Tp})V_{DS} - {V_{DS}}^2 \right] \\
                                       & \text{Resistore controllato da tensione}
                                  \end{aligned}
                              \end{equation*}
                        \item \textbf{Zona di Saturazione (Pinch-off):}
                              \begin{equation*}
                                  \begin{aligned}
                                       & V_{GS} < V_{Tp}, \quad V_{SD} > V_{OV}, \quad V_{GD} > V_{TP} \\
                                       & \text{Generatore di corrente controllato}
                                  \end{aligned}
                              \end{equation*}
                    \end{enumerate}


                \end{minipage}
            };
            %------------ Box 12 Header ---------------------
            \node[fancytitle, right=10pt] at (box.north west) {\color{white}nMOS e pMOS};
        \end{tikzpicture}

        %%%%%%%%%%%%%%%%%%%%%%%%%%%%%%%%%%%%%%%%%%%%%%%%%%%%%%%%%%%%%%%%%%%%


    } %fine resizebox
    \columnbreak
    %------------ cMOS ---------------
    \label{cMOS}
    \begin{tikzpicture}
        \node [mybox] (box){%
            \begin{minipage}{0.3\textwidth}

                \noindent \textbf{cMOS (Inverter):}
                \begin{enumerate}
                    \item \textbf{Ingresso Alto (logica `1`):}
                          \text{rete PULL-UP(pMOS ON SAT, nMOS OFF)}
                          \begin{align*}
                               & \text{nMOS: } V_{GS} > V_{th} \text{ (Conduzione)} \\
                               & \text{pMOS: } V_{GS} > V_{TP} \text{ (Cutoff)}     \\
                               & \text{Uscita: } \text{GND (logica `0`)}
                          \end{align*}
                    \item \textbf{Ingresso Basso (logica `0`):}
                          \text{rete PULL-DOWN(nMOS ON SAT, pMOS OFF)}
                          \begin{align*}
                               & \text{nMOS: } V_{GS} < V_{th} \text{ (Cutoff)}     \\
                               & \text{pMOS: } V_{GS} < V_{TP} \text{ (Conduzione)} \\
                               & \text{Uscita: } \text{VDD (logica `1`)}
                          \end{align*}
                    \item \textbf{Zona intermedia: entrambi in saturazione}
                          \color{red} [.....]\color{black}
                \end{enumerate}

                [grafico delle caratteristiche] \\
                cMOS


            \end{minipage}
        };
        %------------ cMOS Header ---------------------
        \node[fancytitle, right=10pt] at (box.north west) {\color{white}cMOS};
    \end{tikzpicture}

    %------------ AND/nAND ---------------
    \label{AND/nAND}
    \begin{tikzpicture}
        \node [mybox] (box){%
            \begin{minipage}{0.3\textwidth}
                \noindent \textbf{AND (AND/nAND):}
                \begin{enumerate}
                    \item \textbf{Entrata:}
                          \begin{equation*}
                              \begin{aligned}
                                   & A = 0, \quad B = 0 \\
                                   & \text{Uscita: } 0
                              \end{aligned}
                          \end{equation*}

                    \item
                          \begin{equation*}
                              \begin{aligned}
                                   & A = 0, \quad B = 1 \\
                                   & \text{Uscita: } 0
                              \end{aligned}
                          \end{equation*}

                    \item
                          \begin{equation*}
                              \begin{aligned}
                                   & A = 1, \quad B = 0 \\
                                   & \text{Uscita: } 0
                              \end{aligned}
                          \end{equation*}

                    \item
                          \begin{equation*}
                              \begin{aligned}
                                   & A = 1, \quad B = 1 \\
                                   & \text{Uscita: } 1
                              \end{aligned}
                          \end{equation*}
                \end{enumerate}

            \end{minipage}
        };
        %------------ AND/nAND Header ---------------------
        \node[fancytitle, right=10pt] at (box.north west) {\color{white}AND/nAND};
    \end{tikzpicture}

    %------------ OR/nOR ---------------
    \label{OR/nOR}
    \begin{tikzpicture}
        \node [mybox] (box){%
            \begin{minipage}{0.3\textwidth}
                \noindent \textbf{OR (nOR):}
                \begin{enumerate}
                    \item \textbf{Entrata:}
                          \begin{equation*}
                              \begin{aligned}
                                   & A = 0, \quad B = 0 \\
                                   & \text{Uscita: } 0
                              \end{aligned}
                          \end{equation*}

                    \item
                          \begin{equation*}
                              \begin{aligned}
                                   & A = 0, \quad B = 1 \\
                                   & \text{Uscita: } 1
                              \end{aligned}
                          \end{equation*}

                    \item
                          \begin{equation*}
                              \begin{aligned}
                                   & A = 1, \quad B = 0 \\
                                   & \text{Uscita: } 1
                              \end{aligned}
                          \end{equation*}

                    \item
                          \begin{equation*}
                              \begin{aligned}
                                   & A = 1, \quad B = 1 \\
                                   & \text{Uscita: } 1
                              \end{aligned}
                          \end{equation*}
                \end{enumerate}

            \end{minipage}
        };
        %------------ OR/nOR Header ---------------------
        \node[fancytitle, right=10pt] at (box.north west) {\color{white}OR/nOR};
    \end{tikzpicture}

    %------------ XOR/nXOR ---------------
    \label{XOR/nXOR}
    \begin{tikzpicture}
        \node [mybox] (box){%
            \begin{minipage}{0.3\textwidth}
                \noindent \textbf{XOR (nXOR):}
                \begin{enumerate}
                    \item \textbf{Entrata:}
                          \begin{equation*}
                              \begin{aligned}
                                   & A = 0, \quad B = 0 \\
                                   & \text{Uscita: } 0
                              \end{aligned}
                          \end{equation*}

                    \item
                          \begin{equation*}
                              \begin{aligned}
                                   & A = 0, \quad B = 1 \\
                                   & \text{Uscita: } 1
                              \end{aligned}
                          \end{equation*}

                    \item
                          \begin{equation*}
                              \begin{aligned}
                                   & A = 1, \quad B = 0 \\
                                   & \text{Uscita: } 1
                              \end{aligned}
                          \end{equation*}

                    \item
                          \begin{equation*}
                              \begin{aligned}
                                   & A = 1, \quad B = 1 \\
                                   & \text{Uscita: } 0
                              \end{aligned}
                          \end{equation*}
                \end{enumerate}

            \end{minipage}
        };
        %------------ XOR/nXOR Header ---------------------
        \node[fancytitle, right=10pt] at (box.north west) {\color{white}XOR/nXOR};
    \end{tikzpicture}

    %------------ Box 17 ---------------
    \label{Box 17}
    \begin{tikzpicture}
        \node [mybox] (box){%
            \begin{minipage}{0.3\textwidth}
                \dummytext{Content for Box 17}
            \end{minipage}
        };
        %------------ Box 17 Header ---------------------
        \node[fancytitle, right=10pt] at (box.north west) {\color{white}Box 17};
    \end{tikzpicture}


    %------------ Box 18 ---------------
    \label{Box 18}
    \begin{tikzpicture}
        \node [mybox] (box){%
            \begin{minipage}{0.3\textwidth}
                \dummytext{Content for Box 18}
            \end{minipage}
        };
        %------------ Box 18 Header ---------------------
        \node[fancytitle, right=10pt] at (box.north west) {\color{white}Box 18};
    \end{tikzpicture}

    %------------ Box 19 ---------------
    \label{Box 19}
    \begin{tikzpicture}
        \node [mybox] (box){%
            \begin{minipage}{0.3\textwidth}
                \dummytext{Content for Box 19}
            \end{minipage}
        };
        %------------ Box 19 Header ---------------------
        \node[fancytitle, right=10pt] at (box.north west) {\color{white}Box 19};
    \end{tikzpicture}

    %------------ Box 20 ---------------
    \label{Box 20}
    \begin{tikzpicture}
        \node [mybox] (box){%
            \begin{minipage}{0.3\textwidth}
                \dummytext{Content for Box 20}
            \end{minipage}
        };
        %------------ Box 20 Header ---------------------
        \node[fancytitle, right=10pt] at (box.north west) {\color{white}Box 20};
    \end{tikzpicture}

    %------------ Box 21 ---------------
    \label{Box 21}
    \begin{tikzpicture}
        \node [mybox] (box){%
            \begin{minipage}{0.3\textwidth}
                \dummytext{Content for Box 21}
            \end{minipage}
        };
        %------------ Box 21 Header ---------------------
        \node[fancytitle, right=10pt] at (box.north west) {\color{white}Box 21};
    \end{tikzpicture}

    %------------ Box 22 ---------------
    \label{Box 22}
    \begin{tikzpicture}
        \node [mybox] (box){%
            \begin{minipage}{0.3\textwidth}
                \dummytext{Content for Box 22}
            \end{minipage}
        };
        %------------ Box 22 Header ---------------------
        \node[fancytitle, right=10pt] at (box.north west) {\color{white}Box 22};
    \end{tikzpicture}

    %------------ Box 23 ---------------
    \label{Box 23}
    \begin{tikzpicture}
        \node [mybox] (box){%
            \begin{minipage}{0.3\textwidth}
                \dummytext{Content for Box 23}
            \end{minipage}
        };
        %------------ Box 23 Header ---------------------
        \node[fancytitle, right=10pt] at (box.north west) {\color{white}Box 23};
    \end{tikzpicture}

\end{multicols*}
\end{document}