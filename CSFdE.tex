\documentclass[dvipsnames]{article}
\usepackage[a4paper, landscape]{geometry}
\usepackage{url}
\usepackage{multicol} 
\usepackage{amsmath}
\usepackage{esint}
\usepackage{amsfonts}
\usepackage{tikz}
\usetikzlibrary{decorations.pathmorphing}
\usepackage{caption}
\usepackage{amsmath,amssymb}
\usepackage{bbding}
\usepackage{pifont}
\usepackage{wasysym}
\usepackage{colortbl}
\usepackage{xcolor, soul} %colors and \hl
\usepackage{mathtools} 
\usepackage{amsmath,amssymb}
\usepackage[shortlabels]{enumitem}
\usepackage{fontawesome}
\usepackage[american,european, siunitx]{circuitikz} %https://grex99.gitlab.io/circuitgui/
\usepackage{graphicx}
\usepackage{microtype}

\makeatletter

\newcommand*\bigcdot{\mathpalette\bigcdot@{.5}}
\newcommand*\bigcdot@[2]{\mathbin{\vcenter{\hbox{\scalebox{#2}{$\m@th#1\bullet$}}}}}
\makeatother

\definecolor{CustomYellow}{RGB}{233,215,0}

\title{Cheat Sheet}
\usepackage[brazilian]{babel}
\usepackage[utf8]{inputenc}

\advance\topmargin-.8in
\advance\textheight3in
\advance\textwidth3in
\advance\oddsidemargin-1.5in
\advance\evensidemargin-1.5in
\parindent0pt
\parskip2pt
%------------ COMMANDS ---------------
\newcommand{\hr}{\centerline{\rule{3.5in}{1pt}}}
%\colorbox[HTML]{e4e4e4}{\makebox[\textwidth-2\fboxsep][l]{texto}


%------------ CONDITIONAL FORMATTING ---------------
\newcommand\conform{}
%\newcommand\conform{S}
\newcommand\dummytext[1]{\if \conform S {#1} \else \vspace{5cm} \hspace{5cm} \fi}

%---------------------------------------------------
\begin{document}

%\begin{center}{\huge{\textbf{Formulario esame elettrotecnica}}}\\
%\end{center}
\begin{multicols*}{3}

    \tikzstyle{mybox} = [draw=CustomYellow, fill=white, very thick,
    rectangle, rounded corners, inner sep=10pt, inner ysep=10pt]
    \tikzstyle{fancytitle} =[fill=CustomYellow, text=white, font=\bfseries]

    \tikzset{
        miaFreccia/.style={
                ->,
                >=latex,
                line width=1pt,
                shorten <=5mm,
                shorten >=5mm
            }
    }

    %SCALA TRANSITORI

    \resizebox{\columnwidth}{0.9\textheight}{%------------ Box 26 ---------------
        \label{Box 26}
        \begin{tikzpicture}
            \node [mybox] (box){%
                \begin{minipage}{0.3\textwidth}

                    \begin{enumerate}
                        \item Per \colorbox{yellow}{$t \to 0^-$},
                              \begin{enumerate}
                                  \item \textbf{calcolare variabile di stato prima dell’inizio del transitorio}
                                  \item In questa fase il \textbf{\color{orange}condensatore\color{black}}/\textbf{\color{Fuchsia}induttore\color{black}                                   \color{black}} si comporta come \textbf{\color{orange}circuito aperto\color{black}}/\textbf{\color{Fuchsia}cortocircuito\color{black}                                   \color{black}}
                                  \item Sfrutterò nella fase 2 la continuitá della variabile di stato
                              \end{enumerate}

                        \item Per \colorbox{Dandelion}{\(t \to 0^+\) (per var. \textbf{NON} di stato es. $v_x, i_x$) },
                              \begin{enumerate}
                                  \item (Eventuale chiusura interruttore)
                                  \item \textbf{Sfrutto continuitá variabile di stato}: \\ \(\color{orange}v_C(t_0^-)=v_C(t_0^+)\)\color{black} / \color{Fuchsia}\(i_L(t_0^-)= i_L(t_0^+)\)\color{black}
                                  \item \textbf{Sostituisco al transitorio} GENERATORE IDEALE DI \color{orange}\textbf{TENSIONE}\color{black} / \color{Fuchsia}\textbf{CORRENTE }\color{black} con \ul{\textbf{valore pari alla variabile di stato appena calcolata}} $$E=V_C(t \to 0^-) \quad I=I_L(t \to 0^-)$$
                              \end{enumerate}

                        \item Per \colorbox{BurntOrange}{\(t \to \infty\)} / \colorbox{BurntOrange}{\(t>0\)} :
                              \begin{enumerate}
                                  \item \textbf{Soluzione di tipo esponenziale}
                                        \begin{enumerate}
                                            \item Formule variabili di stato:
                                                  \begin{align*}
                                                      V_C(t) & = V_{C_{\infty}}+\left[V_{C}(0)-V_{C_\infty}\right] e^{-\frac{t}{\tau}} \\
                                                      I_L(t) & = I_{L_{\infty}}+\left[I_{L}(0)-I_{L_\infty}\right] e^{-\frac{t}{\tau}}
                                                  \end{align*}

                                            \item Formule per le grandezze \textbf{non di stato}:
                                                  \begin{align*}
                                                      I_C(t) & = I_{C_\infty}+[I_C(\color{red}0^+\color{black})-I_{C_\infty}]e^{\frac{-t}{\tau}} \\
                                                      V_L(t) & = V_{L_\infty}+[V_L(\color{red}0^+\color{black})-V_{L_\infty}]e^{\frac{-t}{\tau}}
                                                  \end{align*}

                                            \item Qui, siamo \textbf{\color{red}ancora a regime\color{black}}: il \textbf{\color{orange}condensatore\color{black}}/\textbf{\color{Fuchsia}induttore\color{black}                                   \color{black}} si comporta come \textbf{\color{orange}circuito aperto\color{black}}/\textbf{\color{Fuchsia}cortocircuito\color{black}                                   \color{black}}
                                            \item Cerco la variabile di stato per \(t \to \infty\)
                                            \item Cerco \(\tau\):
                                                  \begin{enumerate}
                                                      \item Mi serve \(R_{\text{eq}}\) ai morsetti di dove c’é transitorio
                                                      \item \textbf{\color{brown}Spengo generatori non pilotati\color{black}}
                                                      \item uso \color{teal} \textbf{generatore sonda (c.g.)} - cerco corrente che passa sul ramo della sonda in funzione di \(V_S\):  \(?\rightarrow I_S(V_S)\)

                                                            \[
                                                                R_{\text{eq}}=\frac{V_S}{I_S(V_S)}
                                                            \]
                                                            \color{black}
                                                      \item Calcolo  \(\tau\):

                                                            \[
                                                                \tau = C \cdot R_{\text{eq}} = \frac{L}{R_{\text{eq}}}
                                                            \]
                                                  \end{enumerate}
                                        \end{enumerate}


                              \end{enumerate}
                    \end{enumerate}
                \end{minipage}
            };
            %------------ Box 26 Header ---------------------
            \node[fancytitle, right=10pt] at (box.north west) {\color{white}Procedimento transitori:};
        \end{tikzpicture}}


    %------------ Box 27 ---------------
    \label{Box 27}
    \begin{tikzpicture}
        \node [mybox] (box){%
            \begin{minipage}{0.3\textwidth}
                \begin{enumerate}
                    \item Traccio asintoto
                    \item Sfrutto \textbf{proprietá dell’esponenziale}: tangente al grafico in \(t=0\) interseca il valore asintotico dopo \(\Delta t = \tau\)
                    \item Dopo \(t=5\tau\) la funzione assume valore asintotico
                \end{enumerate}

            \end{minipage}
        };
        %------------ Box 27 Header ---------------------
        \node[fancytitle, right=10pt] at (box.north west) {\color{white}Grafico};
    \end{tikzpicture}



    %------------ Box 4: Resistenze in parallelo ---------------
    \label{Box 4}
    \begin{tikzpicture}
        \node [mybox] (box){%
            \begin{minipage}{0.3\textwidth}

                \textbf{Resistenze in parallelo:}

                \begin{enumerate}
                    \item \textbf{Caso con 2 resistenze:}
                          $$R_{eq} = \frac{R_1 \cdot R_2}{R_1 + R_2}$$

                    \item \textbf{Caso generale (n resistenze):}
                          $$\frac{1}{R_{eq}} = \sum_{i=1}^{n} \frac{1}{R_i}$$
                \end{enumerate}

                \vspace{0.3cm}
                \colorbox{red!20}{\parbox{0.95\linewidth}{
                        \textbf{\color{red}$\triangle$ NOTA IMPORTANTE - Tensioni di alimentazione}

                        Le tensioni fornite dalle alimentazioni sono le \textbf{massime e minime} possibili nel circuito.

                        \vspace{0.1cm}
                        \textbf{I NODI della rete NON possono mai avere tensioni:}
                        \begin{itemize}
                            \item Più alte di $V_{\text{max}}$ (alimentazione massima)
                            \item Più basse di $V_{\text{min}}$ (alimentazione minima)
                        \end{itemize}

                        \vspace{0.1cm}
                        \colorbox{yellow!50}{\parbox{0.9\linewidth}{\footnotesize
                                \textbf{ATTENZIONE:} Questo vale per le tensioni dei \textbf{NODI} (riferite a massa).

                                Le \textbf{cadute di tensione} (misurate tra due nodi diversi) \underline{possono} superare questi limiti!
                            }}

                        \vspace{0.1cm}
                        \textbf{Uso pratico:} Fondamentale quando si fanno ipotesi sullo stato dei diodi (ON/OFF). Se un'ipotesi porta un nodo oltre $V_{\text{max}}$ o sotto $V_{\text{min}}$, l'ipotesi è \textbf{sbagliata}.
                    }}

            \end{minipage}
        };
        %------------ Box 4 Header ---------------------
        \node[fancytitle, right=10pt] at (box.north west) {\color{white}Resistenze e Alimentazioni};
    \end{tikzpicture}

    
    %------------ Box 3: Diodi ---------------
    \label{Box 3}
    \begin{tikzpicture}
        \node [mybox] (box){%
            \begin{minipage}{0.3\textwidth}

                \textbf{1. Diodo normale:}

                \vspace{0.2cm}
                \begin{center}
                    \begin{circuitikz}[scale=0.9]
                        % Diodo ON
                        \draw (0,2) to[D, v=$V_D$, color=green!70!black, line width=1.2pt] (2,2);
                        \node[anchor=south] at (1,2.2) {\color{green!70!black}\textbf{ON}};
                        \node[anchor=north] at (1,1.3) {$V_D > 0.7\text{V}$};
                        \node[anchor=north] at (1,0.8) {$\Rightarrow V_D = 0.7\text{V}$};

                        % Diodo OFF
                        \draw (3.5,2) to[D, v=$V_D$, color=gray, line width=1.2pt] (5.5,2);
                        \node[anchor=south] at (4.5,2.2) {\color{gray}\textbf{OFF}};
                        \node[anchor=north] at (4.5,1.3) {$V_D < 0.7\text{V}$};
                        \node[anchor=north] at (4.5,0.8) {$\Rightarrow$ aperto};
                    \end{circuitikz}
                \end{center}

                \vspace{0.3cm}
                \textbf{2. Diodo Zener:}

                \vspace{0.2cm}
                \begin{center}
                    \begin{circuitikz}[scale=0.8]
                        % Conduzione diretta
                        \draw (0,2.5) to[zD, v=$V_D$, color=green!70!black, line width=1.2pt] (2,2.5);
                        \node[anchor=south] at (1,2.7) {\color{green!70!black}\textbf{Diretta}};
                        \node[anchor=north] at (1,1.8) {$V_D > 0.7\text{V}$};
                        \node[anchor=north] at (1,1.2) {$V_D = 0.7\text{V}$};

                        % Breakdown
                        \draw (3.5,2.5) to[zD, v=$V_D$, color=red!70!black, line width=1.2pt] (5.5,2.5);
                        \node[anchor=south] at (4.5,2.7) {\color{red!70!black}\textbf{Breakdown}};
                        \node[anchor=north] at (4.5,1.8) {$V_D < -V_Z$};
                        \node[anchor=north] at (4.5,1.2) {\color{red}$V_D = -V_Z$};
                    \end{circuitikz}
                \end{center}

                \colorbox{yellow!30}{\parbox{0.9\linewidth}{\small \textbf{ATTENZIONE:} In breakdown, la tensione $V_D = -V_Z$ ha polarità \textbf{\color{red}opposta} rispetto ai $+0.7\text{V}$ della conduzione diretta!}}

                \vspace{0.2cm}
                \textbf{3. Configurazioni in serie:}

                \begin{center}
                    \begin{circuitikz}[scale=0.7]
                        % Serie discordi
                        \draw (0,1.5) to[D] (1.5,1.5) to[D, invert] (3,1.5);
                        \node[anchor=south] at (1.5,1.7) {\small Discordi};
                        \node[anchor=north] at (1.5,1.0) {$\Rightarrow$ \textbf{APERTO}};

                        % Serie concordi
                        \draw (4,1.5) to[D] (5.5,1.5) to[D] (7,1.5);
                        \node[anchor=south] at (5.5,1.7) {\small Concordi};
                        \node[anchor=north] at (5.5,1.0) {$V_\gamma = \sum V_{\gamma,i}$};
                    \end{circuitikz}
                \end{center}

                \vspace{0.2cm}
                \colorbox{cyan!20}{\parbox{0.9\linewidth}{\footnotesize
                        \textbf{$\bigstar$ TRUCCO PRATICO - Verifica stato diodo:}

                        Quando sei \textbf{in un intorno della soglia} ($V_D \approx 0.7V$, anche infinitesimamente superiore), le \textbf{correnti sono molto basse}.

                        $\Rightarrow$ Per verificare se il diodo si accende puoi \textbf{ignorare le resistenze in serie} (I $\approx$ 0 $\Rightarrow$ $\Delta V_R \approx$ 0).

                        \textbf{Uso nei transitori:} A fine esercizio, verifica che l'ipotesi sul diodo (ON/OFF) resti valida in:
                        \begin{itemize}[leftmargin=*, itemsep=0pt]
                            \item $\hat{T}^-$ (istante prima della transizione)
                            \item $\hat{T}^+$ (istante dopo della transizione)
                            \item $t \to \infty$ (regime)
                        \end{itemize}
                    }}

            \end{minipage}
        };
        %------------ Box 3 Header ---------------------
        \node[fancytitle, right=10pt] at (box.north west) {\color{white}Diodi};
    \end{tikzpicture}

    %%%%%%%%%%%%%%%%%%%%%%%%%%%%%%%%%%%%%%%%%%%%%%%%%%%%%%%%%%%%%%%%%%%%




    %%%%%%%%%%%%%%%%%%%%%%%%%%%%%%%%%%%%%%%%%%%%%%%%%%%%%%%%%%%%%%%%%%%%





    %------------ Box 5: Capacità - Formule fondamentali ---------------------
    \label{Box 5}
    \begin{tikzpicture}
        \node [mybox] (box){%
            \begin{minipage}{0.3\textwidth}

                \textbf{1. Tensione del condensatore:}
                $$V_C(t) = V_C(0^+) + \left[V_C(\color{orange}\infty^*\color{black}) - V_C(0^+)\right] \left(1 - e^{-\frac{t}{\tau}}\right)$$
                \small
                $V_C(0^+)$: iniziale; $V_C(\color{orange}\infty^*\color{black})$: a regime; \colorbox{yellow!30}{$\color{orange}\infty^* \neq \infty$}

                \vspace{0.2cm}
                \textbf{2. Corrente:} $I_C(t) = C \frac{dV_C(t)}{dt}$

                \small \textbf{\color{blue}Proprietà:}
                La \textbf{\color{red}corrente} varia \textbf{\color{red}istantaneamente}; La \textbf{\color{green!50!black}tensione} NON commuta: $V_C(t_0^-) = V_C(t_0^+)$

                \vspace{0.25cm}
                \colorbox{Dandelion!40}{\parbox{0.95\linewidth}{\small
                        \textbf{$\bigstar$ REGOLA D'ORO - A REGIME}

                        \vspace{0.1cm}
                        A regime ($t \to \infty$): $\frac{dV_C}{dt} = 0$ $\Rightarrow$ $\boxed{I_C = 0}$

                        \textbf{Condensatore = CIRCUITO APERTO}

                        \vspace{0.15cm}
                        \colorbox{white}{\parbox{0.9\linewidth}{\small
                                \textbf{Per calcolare $V_C(\infty)$:}
                                \begin{enumerate}[itemsep=2pt]
                                    \item Sostituisci C con \textbf{circuito aperto}
                                    \item Risolvi il circuito semplificato
                                    \item Calcola la tensione nel punto dove c'era C
                                \end{enumerate}

                                \vspace{0.1cm}
                                \textbf{Es:} $V \xrightarrow{R_1} \bullet \xrightarrow{R_2} \text{GND}$ + C $\parallel$ $R_2$

                                $\Rightarrow$ $V_C(\infty) = V \frac{R_2}{R_1+R_2}$ (partitore)
                            }}
                    }}

                \vspace{0.2cm}
                \textbf{3. Ripple:} $\Delta V_{out} = V_{picco} \frac{\Delta T}{\tau} = V_{picco} \frac{T}{f \cdot \tau}$

                \vspace{0.2cm}
                \textbf{4. Comportamento fisico} \small ($Q = C \cdot V$; $I = C \frac{dV}{dt}$)

                \vspace{0.15cm}
                \colorbox{green!15}{\parbox{0.95\linewidth}{\small
                        \textbf{\color{green!50!black}CARICA} ($\frac{dV_C}{dt} > 0$): Corrente \textbf{ENTRA} ($I_C > 0$)

                        Il condensatore accumula energia; $V_C \uparrow$
                        \begin{center}
                            \begin{circuitikz}[scale=0.55]
                                \draw (0,0) to[C=$C$, v=$V_C$] (0,2.5);
                                \draw[->, thick, color=green!70!black, line width=1.5pt] (-0.9,1.2) -- (-0.15,1.2);
                                \node[anchor=east, color=green!70!black] at (-0.9,1.2) {\small $I_C > 0$};
                                \node[anchor=south] at (0,2.8) {\small Corrente ENTRA};
                            \end{circuitikz}
                        \end{center}
                    }}

                \vspace{0.15cm}
                \colorbox{red!15}{\parbox{0.95\linewidth}{\small
                        \textbf{\color{red!70!black}SCARICA} ($\frac{dV_C}{dt} < 0$): Corrente \textbf{ESCE} ($I_C < 0$)

                        Il condensatore rilascia energia; $V_C \downarrow$
                        \begin{center}
                            \begin{circuitikz}[scale=0.55]
                                \draw (0,0) to[C=$C$, v=$V_C$] (0,2.5);
                                \draw[->, thick, color=red!70!black, line width=1.5pt] (0.15,1.2) -- (0.9,1.2);
                                \node[anchor=west, color=red!70!black] at (0.9,1.2) {\small $I_C < 0$};
                                \node[anchor=south] at (0,2.8) {\small Corrente ESCE};
                            \end{circuitikz}
                        \end{center}
                    }}

                \vspace{0.1cm}
                \colorbox{yellow!20}{\parbox{0.95\linewidth}{\small
                        \textbf{Regola:} $V_C \uparrow$ $\Rightarrow$ CARICA; $V_C \downarrow$ $\Rightarrow$ SCARICA; segno $I_C$ indica verso
                    }}

            \end{minipage}
        };
        %------------ Box 5 Header ---------------------
        \node[fancytitle, right=10pt] at (box.north west) {\color{white}Capacità: Formule e Comportamento};
    \end{tikzpicture}

    %%%%%%%%%%%%%%%%%%%%%%%%%%%%%%%%%%%%%%%%%%%%%%%%%%%%%%%%%%%%%%%%%%%%




    %------------ Box 6: Transitori multipli ---------------
    \label{Box 6}
    \begin{tikzpicture}
        \node [mybox] (box){%
            \begin{minipage}{0.3\textwidth}

                \textbf{Formula tempo centrale $\hat{T}$:}
                $$V_C(\hat{T}) = V_C(0^+)_{\hat{T}} + \left[V_C(\infty^*) - V_C(0^+)_{\hat{T}}\right] \left(1 - e^{-\frac{\hat{T}}{\tau}}\right)$$

                \vspace{0.2cm}
                \colorbox{Dandelion!30}{\parbox{0.95\linewidth}{\small
                        \textbf{Prassi: segnale rettangolare}

                        \underline{salita} $\to$ \underline{plateau} $\to$ \underline{discesa}
                    }}

                \vspace{0.25cm}
                \textbf{Procedimento step-by-step:}
                \begin{enumerate}[leftmargin=*, itemsep=3pt]
                    \item \textbf{\color{green!50!black}FASE 1 - Salita}
                          \begin{itemize}[itemsep=1pt]
                              \item Analizza $t = 0^-$ (condizioni iniziali)
                              \item $V_C(0^+)$ per continuità
                              \item Determina stato diodi
                              \item Calcola $V_C(\infty^*)$
                              \item Applica formula con $\tau$
                          \end{itemize}

                    \item \textbf{\color{orange}FASE 2 - Plateau}
                          \begin{itemize}[itemsep=1pt]
                              \item Se durata $\gg 5\tau$: regime
                              \item Se durata $< 5\tau$: calcola $V_C$ fine
                              \item Verifica diodi (Box 7)
                          \end{itemize}

                    \item \textbf{\color{red!70!black}FASE 3 - Discesa}
                          \begin{itemize}[itemsep=1pt]
                              \item $V_C(0^+) = V_C(\text{fine plateau})$
                              \item Ridetermina stato diodi
                              \item Nuovo $V_C(\infty^*)$
                              \item Applica formula
                          \end{itemize}
                \end{enumerate}

                \vspace{0.2cm}
                \begin{center}
                    \begin{tikzpicture}[scale=0.6]
                        % Segnale di ingresso
                        \draw[->] (0,0) -- (8,0) node[right] {$t$};
                        \draw[->] (0,0) -- (0,2.5) node[above] {$V_{in}$};
                        \draw[thick, green!50!black] (0,0) -- (0.5,0) -- (0.5,2) -- (3.5,2) -- (3.5,0) -- (8,0);
                        \node at (4,-0.8) {\small Ingresso (rettangolare)};

                        % Risposta Vc(t)
                        \draw[->] (0,-3.5) -- (8,-3.5) node[right] {$t$};
                        \draw[->] (0,-3.5) -- (0,-1) node[above] {$V_C$};
                        % Fase 1
                        \draw[thick, green!50!black, domain=0.5:3.5, samples=50] plot (\x, {-3.5 + 2*(1-exp(-(\x-0.5)))});
                        % Fase 2
                        \draw[thick, orange, domain=3.5:3.8, samples=20] plot (\x, {-1.5 - 0.05*(\x-3.5)});
                        % Fase 3
                        \draw[thick, red!70!black, domain=3.8:7, samples=50] plot (\x, {-1.52 - 1.7*(1-exp(-(\x-3.8)))});

                        \node[green!50!black] at (2,-4.2) {\small Fase 1};
                        \node[orange] at (3.65,-4.2) {\small F.2};
                        \node[red!70!black] at (5.5,-4.2) {\small Fase 3};

                        \draw[dashed] (0.5,0) -- (0.5,-3.5);
                        \draw[dashed] (3.5,0) -- (3.5,-3.5);
                        \node at (4,-4.8) {\small Risposta $V_C(t)$};
                    \end{tikzpicture}
                \end{center}

            \end{minipage}
        };
        %------------ Box 6 Header ---------------------
        \node[fancytitle, right=10pt] at (box.north west) {\color{white}Transitori con gradini multipli};
    \end{tikzpicture}

    %%%%%%%%%%%%%%%%%%%%%%%%%%%%%%%%%%%%%%%%%%%%%%%%%%%%%%%%%%%%%%%%%%%%





    %------------ Box 7: Verifica ipotesi diodi ---------------
    \label{Box 7}
    \begin{tikzpicture}
        \node [mybox] (box){%
            \begin{minipage}{0.3\textwidth}

                \colorbox{red!20}{\parbox{0.95\linewidth}{\small
                        \textbf{\color{red}$\triangle$ VERIFICA FONDAMENTALE}

                        Verifica ipotesi diodo (ON/OFF) rimanga valida per tutto il transitorio
                    }}

                \vspace{0.2cm}
                \colorbox{cyan!15}{\parbox{0.95\linewidth}{\small
                        \textbf{\color{blue}FASE 0: Metodo intuitivo}

                        \textbf{Regola:} $I$ scorre da $V_{+}$ a $V_{-}$

                        1) $V_C(0^+)$ continuità
                        2) Trova $V_{\max}$
                        3) $I$ va da $V_{\max}$ a $V_{\min}$
                        4) Compatibile con diodo?
                        5) No $\Rightarrow$ cambia stato

                        \vspace{0.15cm}
                        \begin{center}
                            \begin{circuitikz}[scale=0.65]
                                \draw (0,0) node[ground]{} to[V, v=$V_{in}$] (0,2.5)
                                      to[D, l=$D$, color=red] (2.5,2.5)
                                      to[R, l=$R$] (5,2.5)
                                      to[C, l=$C$, v<=$V_C{=}5V$] (5,0) -- (0,0);
                                \node[anchor=south, color=blue] at (0,2.8) {\small $0V$};
                                \node[anchor=south, color=red!70!black] at (5,2.8) {\small \textbf{$5V$}};
                                \draw[->, very thick, color=orange, line width=1.2pt] (4.5,3.6) -- (0.5,3.6);
                                \node[anchor=south, color=orange] at (2.5,3.8) {\small $I$ (5V$\to$0V)};
                                \node[anchor=north, align=center] at (2.5,-0.5) {\small \textbf{Contraddizione!} $I$ va ← \\
                                \small ma D conduce solo → \\
                                \small $\Rightarrow$ \textbf{D OFF}};
                            \end{circuitikz}
                        \end{center}
                    }}

                \vspace{0.2cm}
                \begin{enumerate}[leftmargin=*, itemsep=2pt]
                    \item Ipotesi (es: D ON)
                    \item Risolvi (ON: gen 0.7V; OFF: aperto)
                    \item Calcola $V_C(t)$
                    \item Verifica $\forall t$:

                          \colorbox{green!10}{\parbox{0.9\linewidth}{\small
                                  \textbf{ON:} $I_D(t) > 0$? No $\to$ errore
                              }}

                          \colorbox{orange!10}{\parbox{0.9\linewidth}{\small
                                  \textbf{OFF:} $V_D(t) < 0.7$V? No $\to$ errore
                              }}

                    \item Se errore: dividi in 2 fasi ($t^*$ cambio), ricalcola
                \end{enumerate}

            \end{minipage}
        };
        %------------ Box 7 Header ---------------------
        \node[fancytitle, right=10pt] at (box.north west) {\color{white}Verifica ipotesi stato diodi};
    \end{tikzpicture}
    %%%%%%%%%%%%%%%%%%%%%%%%%%%%%%%%%%%%%%%%%%%%%%%%%%%%%%%%%%%%%%%%%%%%



    %------------ Box 8 ---------------
    \label{Box 8}
    \begin{tikzpicture}
        \node [mybox] (box){%
            \begin{minipage}{0.3\textwidth}

                \textbf{NOTAZIONE ALIMENTAZIONI}

                \vspace{0.1cm}
                \colorbox{orange!30}{\parbox{0.95\linewidth}{\small
                        \textbf{REGOLA D'ORO:} Il \textbf{trattino} indica SEMPRE il \textbf{polo + del generatore}, sia con tensione positiva che negativa!
                    }}

                \vspace{0.2cm}
                \textbf{Caso 1: $V_{DD} = +1V$ (alimentazione positiva)}

                \begin{center}
                    {\color{black}
                    \begin{circuitikz}[scale=1.0]
                        % Simbolo alimentazione VDD
                        \draw[very thick] (0,0) to[short] (0,0.5);
                        \draw[ultra thick, line width=2pt] (-0.5,0.5) -- (0.5,0.5);
                        \node[anchor=south, font=\small\bfseries] at (0,0.6) {$V_{DD} = +1V$};

                        % Freccia indicazione
                        \draw[<-, ultra thick, color=red, line width=2pt] (-1.2,0.5) -- (-0.55,0.5);
                        \node[anchor=east, color=red, font=\scriptsize\bfseries, align=right] at (-1.25,0.5) {Trattino\\= polo \textbf{+}};

                        % Equivalente circuitale
                        \node[anchor=north, font=\footnotesize, align=center] at (0,-0.2) {Equivale a:};

                        \draw (2.5,-1) node[ground]{} to[V, v=$+1V$, invert] (2.5,0.5);
                        \node[anchor=west, font=\scriptsize] at (3,0.5) {\textbf{+} (polo +)};
                        \node[anchor=west, font=\scriptsize] at (3,-1) {\textbf{--} (GND)};
                    \end{circuitikz}
                    }
                \end{center}

                \vspace{0.05cm}
                {\footnotesize Tensione $+1V$ $\to$ polo + sul trattino, tutto normale}

                \vspace{0.2cm}
                \textbf{Caso 2: $V_{SS} = -1V$ (alimentazione negativa)}

                \begin{center}
                    {\color{black}
                    \begin{circuitikz}[scale=1.0]
                        % Simbolo alimentazione VSS
                        \draw[very thick] (0,0) to[short] (0,0.5);
                        \draw[ultra thick, line width=2pt] (-0.5,0.5) -- (0.5,0.5);
                        \node[anchor=south, font=\small\bfseries] at (0,0.6) {$V_{SS} = -1V$};

                        % Freccia indicazione
                        \draw[<-, ultra thick, color=blue, line width=2pt] (-1.2,0.5) -- (-0.55,0.5);
                        \node[anchor=east, color=blue, font=\scriptsize\bfseries, align=right] at (-1.25,0.5) {Trattino\\= polo \textbf{+}};

                        % Equivalente circuitale
                        \node[anchor=north, font=\footnotesize, align=center] at (0,-0.2) {Equivale a:};

                        \draw (2.5,-1) node[ground]{} to[V, v=$-1V$, invert] (2.5,0.5);
                        \node[anchor=west, font=\scriptsize] at (3,0.5) {\textbf{+} (polo +)};
                        \node[anchor=west, font=\scriptsize] at (3,-1) {\color{black}\textbf{--} (GND)};
                    \end{circuitikz}
                    }
                \end{center}

                \vspace{0.05cm}
                {\footnotesize Tensione $-1V$ $\to$ polo + è comunque sul trattino!}

                \vspace{0.15cm}
                \colorbox{cyan!20}{\parbox{0.95\linewidth}{\footnotesize
                        \textbf{TRUCCO:} Con $V_{SS}=-1V$ puoi ridisegnare il generatore invertendo polarità E segno: diventa $+1V$ con polo + su GND. Utile per evitare tensioni negative nei calcoli.
                    }}

            \end{minipage}
        };
        %------------ Box 8 Header ---------------------
        \node[fancytitle, right=10pt] at (box.north west) {\color{white}Notazione alimentazioni};
    \end{tikzpicture}


    % Separa sezione MOSFET
    \newpage



    %%%%%%%%%%%%%

    %------------ Formazione del Canale nei MOSFET ---------------
    \label{Formazione del Canale nei MOSFET}
    \begin{tikzpicture}
        \node [mybox] (box){%
            \begin{minipage}{0.3\textwidth}



                \begin{enumerate}
                    \item \textbf{Zona OFF (o Cutoff):}
                          \begin{enumerate}
                              \item Non c'è formazione del canale.
                              \item Il dispositivo è spento e non permette il flusso di corrente tra drain e source.
                          \end{enumerate}

                    \item \textbf{Zona Ohmica (o Triodo):}
                          \begin{enumerate}
                              \item Si forma un canale.
                              \item Quando il gate è abbastanza polarizzato (cioè \( V_{GS} > V_{Tn} \) per nMOS o \( V_{GS} < V_{Tp} \) per pMOS), si forma un canale conduttivo tra il drain e il source.
                              \item Il dispositivo si comporta come un \textbf{resistore il cui valore varia in base alla tensione} \( V_{GS} \).
                          \end{enumerate}

                    \item \textbf{Zona di Saturazione (o Pinch-off):}
                          \begin{enumerate}
                              \item Si forma un canale.
                              \item Il canale diventa "strozzato" o "pinched-off" vicino al drain (per il nMOS) o vicino al source (per il pMOS).
                              \item Anche se la tensione \( V_{DS} \) aumenta ulteriormente, la corrente \( I_D \) rimane costante.
                              \item Questo comportamento è \textbf{analogo a quello di un generatore di corrente.}
                          \end{enumerate}
                \end{enumerate}



            \end{minipage}
        };
        %------------ Formazione del Canale nei MOSFET Header ---------------------
        \node[fancytitle, right=10pt] at (box.north west) {\color{white}Formazione del Canale nei MOSFET};
    \end{tikzpicture}

    %%%%%%%%%%%%%%%%%%%%%%%%%%%%%%%%%%%%%%%%%%%%%%%%%%%%%%%%%%%%%%%%%%%%


    %%%%%%%%%%%%%%%%%%%%%%%%%%%%%%%%%%%%%%%%%%%%%%%%%%%%%%%%%%%%%%%%%%%%

    %------------ Box: Simboli e convenzioni nMOS/pMOS ---------------
    \label{Box Simboli MOSFET}
    \begin{tikzpicture}
        \node [mybox] (box){%
            \begin{minipage}{0.3\textwidth}

                \vspace{0.3cm}
                \textbf{\color{blue}nMOS:}

                \begin{center}
                    \begin{circuitikz}[scale=2.0]
                        % nMOS symbol
                        \draw (0,0) node[nmos, anchor=G] (nmos) {};

                        % Labels terminali - vicine al simbolo
                        \node[anchor=east, font=\small] at ($(nmos.G)+(-0.1,0)$) {G};
                        \node[anchor=south, font=\small] at ($(nmos.D)+(0,0.001)$) {D};
                        \node[anchor=north, font=\small] at ($(nmos.S)+(0,-0.001)$) {S};

                        % Freccia VGS - a sinistra con concavità invertita
                        \draw[->, thick, color=red!70!black] ($(nmos.G)+(-0.2,-0.1)$) to[bend right=40] node[midway, left, font=\footnotesize] {$V_{GS}$} ($(nmos.S)+(-0.5,-0.15)$);

                        % Freccia VDS - a destra con concavità invertita
                        \draw[->, thick, color=blue!70!black] ($(nmos.D)+(0.5,-0.15)$) to[bend left=40] node[midway, right, font=\footnotesize] {$V_{DS}$} ($(nmos.S)+(0.3,-0.15)$);

                        % Freccia corrente - molto più a destra e accorciata
                        \draw[->, line width=2pt, color=green!60!black] ($(nmos.D)+(1.3,0.15)$) -- ($(nmos.S)+(1.3,-0.15)$);
                        \node[anchor=west, color=green!60!black, font=\small\bfseries] at ($(nmos.S)+(1.5,-0.05)$) {$I_D$};

                        % Freccia entrante sul terminale Source (che è in alto nel pMOS)
                        \draw[->, ultra thick, color=blue!70!black]  ($(nmos.S)+(-0.2,0.25)$) -- ($(nmos.S)+(0,0.25)$) ;
                    \end{circuitikz}
                \end{center}

                \vspace{0.1cm}
                \colorbox{green!15}{\parbox{0.95\linewidth}{\small
                    \textbf{nMOS:} Gate a sinistra, Drain in alto, Source in basso

                    \textbf{Corrente:} Da Drain $\to$ Source (verso il basso)
                }}

                \vspace{0.4cm}
                \textbf{\color{purple}pMOS:}

                \begin{center}
                    \begin{circuitikz}[scale=2.0]
                        % pMOS symbol (invertito)
                        \draw (0,0) node[pmos, anchor=G] (pmos) {};

                        % Labels terminali - vicine al simbolo
                        \node[anchor=east, font=\small] at ($(pmos.G)+(-0.25,0)$) {G};
                        \node[anchor=south, font=\small] at ($(pmos.D)+(0,0.65)$) {S};
                        \node[anchor=north, font=\small] at ($(pmos.S)+(0,-0.65)$) {D};

                        % Freccia VGS - a sinistra con concavità invertita rispetto a nMOS
                        \draw[->, thick, color=red!70!black] ($(pmos.G)+(-0.2,0.1)$) to[bend right=40] node[midway, left, font=\footnotesize] {$V_{GS}$} ($(pmos.D)+(-0.5,-0.15)$);

                        % Freccia VSD - a destra con concavità invertita rispetto a nMOS
                        \draw[->, thick, color=blue!70!black] ($(pmos.D)+(0.5,0.15)$) to[bend right=40] node[midway, right, font=\footnotesize] {$V_{SD}$} ($(pmos.S)+(0.3,0.15)$);

                        % Freccia corrente - molto più a destra e accorciata
                        \draw[->, line width=2pt, color=purple!70!black] ($(pmos.D)+(1.3,0.15)$) -- ($(pmos.S)+(1.3,-0.15)$);
                        \node[anchor=west, color=purple!70!black, font=\small\bfseries] at ($(pmos.S)+(1.5,-0.05)$) {$I_D$};

                        % Freccia entrante sul terminale Source (che è in alto nel pMOS)
                        \draw[->, ultra thick, color=red!70!black]  ($(pmos.S)+(0,-0.25)$) --  ($(pmos.S)+(-0.2,-0.25)$);
                    \end{circuitikz}
                \end{center}

                \vspace{0.1cm}
                \colorbox{purple!15}{\parbox{0.95\linewidth}{\small
                    \textbf{pMOS:} Gate a sinistra, Source in alto, Drain in basso

                    \textbf{Corrente:} Da Source $\to$ Drain (verso il basso)

                    \textbf{NOTA:} Nel pMOS il source è in alto (invertito rispetto a nMOS)!
                }}

            \end{minipage}
        };
        %------------ Box Simboli MOSFET Header ---------------------
        \node[fancytitle, right=10pt] at (box.north west) {\color{white}Simboli e convenzioni nMOS/pMOS};
    \end{tikzpicture}



    %%%%%%%%%%%%%%%%%%%%%%%%%%%%%%%%%%%%%%%%%%%%%%%%%%%%%%%%%%%%%%%%%%%%

    %------------ Box: Caratteristica I-V nMOS ---------------
    \label{Box Caratteristica nMOS}
    \begin{tikzpicture}
        \node [mybox] (box){%
            \begin{minipage}{0.3\textwidth}


                \begin{center}
                    \begin{tikzpicture}[scale=0.9]
                        % Assi
                        \draw[->] (0,0) -- (5.5,0) node[right] {$V_{DS}$};
                        \draw[->] (0,0) -- (0,4.2) node[above] {$I_D$};

                        % Curve caratteristiche per diversi VGS
                        % VGS1 (più basso)
                        \draw[thick, blue] (0,0) -- (0.5,0.8) .. controls (1,1) and (1.5,1.1) .. (5,1.1);
                        \node[anchor=west, blue, font=\tiny] at (5.1,1.1) {$V_{GS1}$};

                        % VGS2
                        \draw[thick, blue] (0,0) -- (1,1.6) .. controls (1.5,2) and (2,2.1) .. (5,2.1);
                        \node[anchor=west, blue, font=\tiny] at (5.1,2.1) {$V_{GS2}$};

                        % VGS3
                        \draw[thick, blue] (0,0) -- (1.5,2.4) .. controls (2,3) and (2.5,3.1) .. (5,3.1);
                        \node[anchor=west, blue, font=\tiny] at (5.1,3.1) {$V_{GS3}$};

                        % Zona INTERDIZIONE (OFF) - spostata più a sinistra
                        \fill[red!10, opacity=0.3] (-0.6,-0.3) rectangle (-0.1,4.3);
                        \node[rotate=90, font=\tiny\bfseries, red!70!black] at (-0.8,2.3) {INTERDIZIONE};
                        \node[rotate=90, font=\tiny, red!70!black] at (-0.35,1) {$V_{GS}<V_T$};

                        % Linea divisione triodo/saturazione
                        \draw[dashed, thick, orange!70!black] (0,0) -- (2.5,4.2);

                        % Punto di separazione sull'asse X con scritta sotto
                        \draw[orange!70!black, very thick] (1.5,-0.1) -- (1.5,0.1);
                        \node[anchor=north, orange!70!black, font=\tiny] at (1.5,-0.35) {$V_{DS}=V_{GS}-V_T$};

                        % Zona TRIODO (OMICA)
                        \node[font=\footnotesize\bfseries, cyan!70!black] at (0.8,2.8) {TRIODO};
                        \node[font=\tiny, cyan!70!black] at (0.8,2.4) {(OMICA)};

                        % Zona SATURAZIONE - spostata più in alto
                        \node[font=\footnotesize\bfseries, green!60!black] at (3.8,3.8) {SATURAZIONE};
                        \node[font=\tiny, green!60!black] at (3.8,3.5) {$V_{DS}>V_{GS}-V_T$};

                    \end{tikzpicture}
                \end{center}

                \vspace{0.2cm}
                \colorbox{yellow!20}{\parbox{0.95\linewidth}{\small
                    \textbf{Zone di funzionamento:}
                    \begin{itemize}[itemsep=1pt, leftmargin=*]
                        \item \textbf{\color{red!70!black}INTERDIZIONE}: $V_{GS} < V_T$ → $I_D = 0$
                        \item \textbf{\color{cyan!70!black}TRIODO}: $V_{GS} > V_T$ e $V_{DS} < (V_{GS}-V_T)$
                        \item \textbf{\color{green!60!black}SATURAZIONE}: $V_{GS} > V_T$ e $V_{DS} > (V_{GS}-V_T)$
                    \end{itemize}
                }}

            \end{minipage}
        };
        %------------ Box Caratteristica nMOS Header ---------------------
        \node[fancytitle, right=10pt] at (box.north west) {\color{white}Caratteristica I-V nMOS};
    \end{tikzpicture}

    %%%%%%%%%%%%%%%%%%%%%%%%%%%%%%%%%%%%%%%%%%%%%%%%%%%%%%%%%%%%%%%%%%%%

    %------------ Box: Caratteristica I-V pMOS ---------------
    \label{Box Caratteristica pMOS}
    \begin{tikzpicture}
        \node [mybox] (box){%
            \begin{minipage}{0.3\textwidth}
                \begin{center}
                    \begin{tikzpicture}[scale=0.9]
                        % Assi
                        \draw[->] (0,0) -- (5.5,0) node[right] {$|V_{DS}|$};
                        \draw[->] (0,0) -- (0,4.2) node[above] {$I_D$};

                        % Curve caratteristiche per diversi VSG
                        % VSG1 (più basso)
                        \draw[thick, purple] (0,0) -- (0.5,0.8) .. controls (1,1) and (1.5,1.1) .. (5,1.1);
                        \node[anchor=west, purple, font=\tiny] at (5.1,1.1) {$V_{SG1}$};

                        % VSG2
                        \draw[thick, purple] (0,0) -- (1,1.6) .. controls (1.5,2) and (2,2.1) .. (5,2.1);
                        \node[anchor=west, purple, font=\tiny] at (5.1,2.1) {$V_{SG2}$};

                        % VSG3
                        \draw[thick, purple] (0,0) -- (1.5,2.4) .. controls (2,3) and (2.5,3.1) .. (5,3.1);
                        \node[anchor=west, purple, font=\tiny] at (5.1,3.1) {$V_{SG3}$};

                        % Zona INTERDIZIONE (OFF) - spostata più a sinistra
                        \fill[red!10, opacity=0.3] (-0.6,-0.3) rectangle (-0.1,4.3);
                        \node[rotate=90, font=\tiny\bfseries, red!70!black] at (-0.8,2.3) {INTERDIZIONE};
                        \node[rotate=90, font=\tiny, red!70!black] at (-0.35,0.8) {$V_{SG}<|V_T|$};

                        % Linea divisione triodo/saturazione
                        \draw[dashed, thick, orange!70!black] (0,0) -- (2.5,4.2);

                        % Punto di separazione sull'asse X con scritta sotto
                        \draw[orange!70!black, very thick] (1.5,-0.1) -- (1.5,0.1);
                        \node[anchor=north, orange!70!black, font=\tiny] at (1.5,-0.35) {$V_{SD}=V_{SG}-|V_T|$};

                        % Zona TRIODO (OMICA)
                        \node[font=\footnotesize\bfseries, cyan!70!black] at (0.8,2.8) {TRIODO};
                        \node[font=\tiny, cyan!70!black] at (0.8,2.4) {(OMICA)};

                        % Zona SATURAZIONE - spostata più in alto
                        \node[font=\footnotesize\bfseries, green!60!black] at (3.8,3.8) {SATURAZIONE};
                        \node[font=\tiny, green!60!black] at (3.8,3.5) {$V_{SD}>V_{SG}-|V_T|$};

                    \end{tikzpicture}
                \end{center}

                \vspace{0.2cm}
                \colorbox{yellow!20}{\parbox{0.95\linewidth}{\small
                    \textbf{Zone di funzionamento:}
                    \begin{itemize}[itemsep=1pt, leftmargin=*]
                        \item \textbf{\color{red!70!black}INTERDIZIONE}: $V_{SG} < |V_T|$ → $I_D = 0$
                        \item \textbf{\color{cyan!70!black}TRIODO}: $V_{SG} > |V_T|$ e $V_{SD} < (V_{SG}-|V_T|)$
                        \item \textbf{\color{green!60!black}SATURAZIONE}: $V_{SG} > |V_T|$ e $V_{SD} > (V_{SG}-|V_T|)$
                    \end{itemize}
                }}

            \end{minipage}
        };
        %------------ Box Caratteristica pMOS Header ---------------------
        \node[fancytitle, right=10pt] at (box.north west) {\color{white}Caratteristica I-V pMOS};
    \end{tikzpicture}


    \newpage 



    %%%%%%%%%%%%%%%%%%%%%%%%%%%%%%%%%%%%%%%%%%%%%%%%%%%%%%%%%%%%%%%%%%%%

    %------------ Box: nMOS - Metodo operativo ---------------
    \label{Box nMOS}
    \begin{tikzpicture}
        \node [mybox] (box){%
            \begin{minipage}{0.3\textwidth}


                \vspace{0.2cm}
                \textbf{\color{blue}PRIMO CONTROLLO: $V_{GS}$ vs $V_{T}$}

                \begin{enumerate}[leftmargin=*, itemsep=3pt]
                    \item Se \colorbox{red!20}{$V_{GS} < V_T$} $\Rightarrow$ \textbf{MOSFET OFF}
                          \begin{itemize}
                              \item $I_D = 0$ (circuito aperto)
                              \item Non c'è conduzione
                          \end{itemize}

                    \item Se \colorbox{green!20}{$V_{GS} > V_T$} $\Rightarrow$ \textbf{MOSFET ON}
                          \begin{itemize}
                              \item Proseguire al \textbf{SECONDO CONTROLLO}
                          \end{itemize}
                \end{enumerate}

                \vspace{0.2cm}
                \textbf{\color{blue}SECONDO CONTROLLO (solo se ON): $V_{DS}$ vs $(V_{GS} - V_T)$}

                \vspace{0.1cm}
                \colorbox{yellow!20}{\parbox{0.95\linewidth}{
                    \textbf{Tensione di Overdrive:}
                    $$V_{OV} = V_{GS} - V_T$$
                }}

                \vspace{0.15cm}
                \begin{enumerate}[leftmargin=*, itemsep=3pt]
                    \item \textbf{\color{orange}ZONA DI SATURAZIONE}: Se $V_{DS} > (V_{GS} - V_T)$

                          \colorbox{orange!15}{\parbox{0.93\linewidth}{
                              $$I_D = K_n(V_{GS} - V_T)^2$$

                              \vspace{0.1cm}
                              \textbf{Nota:} La corrente dipende \textbf{SOLO da $V_{GS}$}
                          }}

                    \item \textbf{\color{cyan!70!black}ZONA OHMICA (Triodo)}: Se $V_{DS} < (V_{GS} - V_T)$

                          \colorbox{cyan!15}{\parbox{0.93\linewidth}{
                              $$I_D = K_n\left[2(V_{GS} - V_T)V_{DS} - V_{DS}^2\right]$$

                              \vspace{0.1cm}
                              \textbf{Nota:} La corrente dipende da \textbf{$V_{GS}$ E $V_{DS}$}
                          }}

                          \vspace{0.15cm}
                          \colorbox{cyan!10}{\parbox{0.93\linewidth}{\small
                              \textbf{Formula alternativa:}
                              $$I_D = \frac{1}{2} K_n V_{OV} \left(V_{DS} - \frac{V_{DS}^2}{2}\right)$$
                          }}
                \end{enumerate}

                \small \textbf{Direzione corrente:} In nMOS, $I_D$ scorre da \textbf{Drain $\to$ Source}

            \end{minipage}
        };
        %------------ Box nMOS Header ---------------------
        \node[fancytitle, right=10pt] at (box.north west) {\color{white}nMOS - Metodo operativo};
    \end{tikzpicture}

    %%%%%%%%%%%%%%%%%%%%%%%%%%%%%%%%%%%%%%%%%%%%%%%%%%%%%%%%%%%%%%%%%%%%

    %------------ Box: pMOS - Metodo operativo ---------------
    \label{Box pMOS}
    \begin{tikzpicture}
        \node [mybox] (box){%
            \begin{minipage}{0.3\textwidth}



                \vspace{0.2cm}
                \textbf{\color{blue}PROCEDIMENTO OPERATIVO PER pMOS}

                \vspace{0.15cm}
                \colorbox{red!30}{\parbox{0.95\linewidth}{\footnotesize
                    \textbf{\color{red}$\bigstar$ STEP 0 - CONTROLLO POLARITÀ}

                    \vspace{0.05cm}
                    \textbf{Prima di tutto}, verifica che:
                    $$V_S > V_G$$

                    \vspace{0.05cm}
                    Se $V_S \leq V_G$ → \textbf{pMOS OFF} (anche se $|V_{GS}| \geq |V_T|$!)

                    \vspace{0.05cm}
                    \textbf{Motivo:} Il modulo $|V_{GS}|$ nasconde il segno! Potresti avere $|V_{GS}| \geq |V_T|$ ma con polarità sbagliata (es. $V_{GS} > 0$), e il pMOS sarebbe OFF.
                }}

                \vspace{0.2cm}
                \textbf{\color{blue}Step 1: Calcolare $|V_{GS}|$}

                \vspace{0.05cm}
                (solo se hai verificato $V_S > V_G$)

                \vspace{0.15cm}
                \textbf{\color{blue}Step 2: PRIMO CONTROLLO - $|V_{GS}|$ vs $|V_T|$}

                \begin{enumerate}[leftmargin=*, itemsep=3pt]
                    \item Se \colorbox{red!20}{$|V_{GS}| < |V_T|$} $\Rightarrow$ \textbf{MOSFET OFF}
                          \begin{itemize}
                              \item $I_D = 0$ (circuito aperto)
                              \item Non c'è conduzione
                          \end{itemize}

                    \item Se \colorbox{green!20}{$|V_{GS}| > |V_T|$} $\Rightarrow$ \textbf{MOSFET ON}
                          \begin{itemize}
                              \item Calcolare $V_{OV} = |V_{GS}| - |V_T|$
                              \item Proseguire allo \textbf{Step 3}
                          \end{itemize}
                \end{enumerate}

                \vspace{0.2cm}
                \textbf{\color{blue}Step 3: SECONDO CONTROLLO - $|V_{DS}|$ vs $V_{OV}$}

                \vspace{0.1cm}
                \colorbox{yellow!20}{\parbox{0.95\linewidth}{
                    \textbf{Tensione di Overdrive:}
                    $$V_{OV} = |V_{GS}| - |V_T|$$
                }}

                \vspace{0.15cm}
                \begin{enumerate}[leftmargin=*, itemsep=3pt]
                    \item \textbf{\color{orange}ZONA DI SATURAZIONE}: Se $|V_{DS}| > V_{OV}$

                          \colorbox{orange!15}{\parbox{0.93\linewidth}{
                              $$I_D = K_p \cdot V_{OV}^2 = K_p(|V_{GS}| - |V_T|)^2$$

                              \vspace{0.1cm}
                              \textbf{Nota:} La corrente dipende \textbf{SOLO dall'overdrive}
                          }}

                    \item \textbf{\color{cyan!70!black}ZONA OHMICA (Triodo)}: Se $|V_{DS}| < V_{OV}$

                          \colorbox{cyan!15}{\parbox{0.93\linewidth}{
                              $$I_D = K_p\left[2 V_{OV} \cdot |V_{DS}| - |V_{DS}|^2\right]$$

                              \vspace{0.05cm}
                              {\footnotesize dove $V_{OV} = |V_{GS}| - |V_T|$}

                              \vspace{0.1cm}
                              \textbf{Nota:} La corrente dipende da \textbf{$V_{OV}$ E $|V_{DS}|$}
                          }}
                \end{enumerate}

                \vspace{0.2cm}
                \small \textbf{Direzione corrente:} In pMOS, $I_D$ scorre da \textbf{Source $\to$ Drain}

            \end{minipage}
        };
        %------------ Box pMOS Header ---------------------
        \node[fancytitle, right=10pt] at (box.north west) {\color{white}pMOS - Metodo operativo};
    \end{tikzpicture}

    %%%%%%%%%%%%%%%%%%%%%%%%%%%%%%%%%%%%%%%%%%%%%%%%%%%%%%%%%%%%%%%%%%%%


    %------------ Box: Riepilogo nMOS vs pMOS ---------------
    \begin{tikzpicture}
        \node [mybox] (box){%
            \begin{minipage}{0.3\textwidth}

                \small
                \textbf{\color{purple}Grandezze da calcolare per determinare lo stato:}

                \vspace{0.2cm}

                \begin{tabular}{|c|c|}
                    \hline
                    \rowcolor{green!15}
                    \textbf{nMOS} & \textbf{pMOS} \\
                    \hline
                    $V_{GS}$ & $|V_{GS}|$ \\
                    \hline
                    $V_T$ & $|V_T|$ \\
                    \hline
                    $V_{OV} = V_{GS} - V_T$ & $V_{OV} = |V_{GS}| - |V_T|$ \\
                    \hline
                    $V_{DS}$ & $|V_{DS}|$ \\
                    \hline
                \end{tabular}

                \vspace{0.2cm}
                \colorbox{cyan!10}{\parbox{0.95\linewidth}{\footnotesize
                    \textbf{Controlli identici:}
                    \begin{enumerate}[itemsep=1pt, leftmargin=*]
                        \item Se $V_{GS}$ (o $|V_{GS}|$) $<$ $V_T$ (o $|V_T|$) $\Rightarrow$ OFF
                        \item Se ON: confronta $V_{DS}$ (o $|V_{DS}|$) con $V_{OV}$
                    \end{enumerate}

                    \vspace{0.05cm}
                    La \textbf{procedura è identica}, solo che per pMOS si usano i \textbf{valori assoluti}.
                }}

            \end{minipage}
        };
        %------------ Box Riepilogo Header ---------------------
        \node[fancytitle, right=10pt] at (box.north west) {\color{white}Riepilogo: nMOS vs pMOS};
    \end{tikzpicture}


    %------------ MOSFET simmetrici - Source e Drain a runtime ---------------------
    \begin{tikzpicture}
        \node [mybox] (box){%
            \begin{minipage}{0.3\textwidth}

                \small

                \vspace{0.15cm}
                \colorbox{red!20}{\parbox{0.95\linewidth}{\footnotesize
                    \textbf{\color{red}$\bigstar$ MOSFET SIMMETRICI}

                    \vspace{0.05cm}
                    I MOSFET sono dispositivi \textbf{simmetrici}: Source e Drain \textbf{NON sono fissi} ma vengono determinati dalle \textbf{tensioni a runtime}!
                }}

                \vspace{0.15cm}
                \textbf{\underline{Come identificare i terminali negli esercizi:}}

                \vspace{0.1cm}
                \colorbox{yellow!30}{\parbox{0.95\linewidth}{\footnotesize
                    \textbf{GATE} (sempre indicato):

                    • \textbf{nMOS}: gate \textbf{senza pallino}

                    • \textbf{pMOS}: gate \textbf{con pallino} (\textbullet)
                }}

                \vspace{0.15cm}
                \colorbox{cyan!15}{\parbox{0.95\linewidth}{\footnotesize
                    \textbf{SOURCE e DRAIN} (determinati a runtime):

                    Se non indicati esplicitamente nel testo, si determinano in base alle \textbf{tensioni dei nodi}.
                }}

                \vspace{0.15cm}
                \textbf{\underline{Regole per determinare SOURCE:}}

                \vspace{0.1cm}
                \textbf{1. nMOS}

                \vspace{0.05cm}
                \colorbox{green!20}{\parbox{0.95\linewidth}{\footnotesize
                    Il \textbf{SOURCE} è il nodo alla \textbf{tensione più BASSA} tra i due terminali non-gate.

                    \vspace{0.05cm}
                    Il DRAIN è l'altro terminale (tensione più alta).
                }}

                \vspace{0.15cm}
                \textbf{2. pMOS}

                \vspace{0.05cm}
                \colorbox{orange!20}{\parbox{0.95\linewidth}{\footnotesize
                    Il \textbf{SOURCE} è il nodo alla \textbf{tensione più ALTA} tra i due terminali non-gate.

                    \vspace{0.05cm}
                    Il DRAIN è l'altro terminale (tensione più bassa).
                }}

                \vspace{0.2cm}
                \hrule
                \vspace{0.15cm}

                \colorbox{red!15}{\parbox{0.95\linewidth}{\footnotesize
                    \textbf{\color{red}$\bigstar$ ATTENZIONE - Riassegnazione a RUNTIME}

                    \vspace{0.05cm}
                    Durante l'esercizio, le tensioni ai nodi possono \textbf{cambiare}!

                    \vspace{0.05cm}
                    $\Rightarrow$ Source e Drain possono essere \textbf{riassegnati} in base alle nuove tensioni.

                    \vspace{0.05cm}
                    Devi \textbf{verificare quale nodo ha la tensione più alta/bassa} in ogni fase dell'analisi!
                }}

                \vspace{0.15cm}
                \colorbox{purple!15}{\parbox{0.95\linewidth}{\tiny
                    \textbf{Esempio pratico (nMOS):}

                    Inizialmente: Nodo A = 3V, Nodo B = 1V

                    $\Rightarrow$ Source = B (1V, più basso), Drain = A (3V)

                    \vspace{0.03cm}
                    Dopo un transitorio: Nodo A = 0.5V, Nodo B = 2V

                    $\Rightarrow$ Source = A (0.5V, più basso), Drain = B (2V)

                    \vspace{0.03cm}
                    I terminali sono stati \textbf{invertiti}!
                }}

                \vspace{0.15cm}
                \colorbox{cyan!10}{\parbox{0.95\linewidth}{\tiny
                    \textbf{Perché è importante:}

                    • $V_{GS}$ e $V_{DS}$ dipendono da quale terminale è il Source

                    • Per calcolare correttamente le formule, devi identificare Source e Drain correttamente in ogni momento

                    • La zona di funzionamento (saturazione/omica) dipende da $V_{DS}$, quindi dall'identificazione corretta dei terminali
                }}

            \end{minipage}
        };
        %------------ MOSFET simmetrici Header ---------------------
        \node[fancytitle, right=10pt] at (box.north west) {\color{white}MOSFET simmetrici - Source e Drain a runtime};
    \end{tikzpicture}


    %------------ Overdrive e Parametro K ---------------------
    \label{Overdrive}
    \begin{tikzpicture}
        \node [mybox] (box){%
            \begin{minipage}{0.3\textwidth}

                $$K = \frac{1}{2} \mu \cdot C_{OX} \cdot \frac{W}{L}$$

                \noindent Dove:
                \begin{itemize}[itemsep=1pt, leftmargin=*]
                    \item $\mu$ = mobilità dei portatori nel canale
                    \item $C_{OX}$ = capacità specifica dell'ossido
                    \item $W/L$ = dimensioni fisiche del MOSFET \\
                          (Width/Length)
                \end{itemize}

                \vspace{0.2cm}
                \colorbox{orange!20}{\parbox{0.95\linewidth}{\small
                    \textbf{\color{red}$\triangle$ NOTA IMPORTANTE - Fattore 1/2}

                    \vspace{0.1cm}
                    K può essere definito \textbf{SENZA} il fattore $\frac{1}{2}$ al suo interno.

                    In tal caso, le formule delle correnti devono essere \textbf{riadattate}:

                    \vspace{0.1cm}
                    \begin{itemize}[itemsep=2pt]
                        \item \textbf{Saturazione:} \\
                              $I = \frac{K}{2}(V_{GS}-V_T)^2$ invece di $I = K(V_{GS}-V_T)^2$

                        \item \textbf{Omica:} \\
                              $I = K\left[(V_{GS}-V_T)V_{DS} - \frac{V_{DS}^2}{2}\right]$ \\
                              invece di $I = K\left[2(V_{GS}-V_T)V_{DS} - V_{DS}^2\right]$
                    \end{itemize}
                }}

            \end{minipage}
        };
        %------------ Overdrive Header ---------------------
        \node[fancytitle, right=10pt] at (box.north west) {\color{white}Parametro K (Transconduttanza)};
    \end{tikzpicture}


    %------------ MOSFET in parallelo ---------------
    \begin{tikzpicture}
        \node [mybox] (box){%
            \begin{minipage}{0.3\textwidth}

                \small


                \vspace{0.1cm}
                \colorbox{red!20}{\parbox{0.95\linewidth}{\footnotesize
                    \textbf{\color{red}$\bigstar$ CONDIZIONE FONDAMENTALE:}

                    Tutti i \textbf{GATE devono essere in COMUNE} (stessa tensione al gate)
                }}

                \vspace{0.15cm}
                \textbf{\underline{1. MOSFET in PARALLELO}}

                \vspace{0.1cm}
                \colorbox{yellow!30}{\parbox{0.95\linewidth}{\footnotesize
                    • GATE in comune

                    • SOURCE in comune (vengono mantenuti)
                }}

                \vspace{0.1cm}
                \textbf{\color{purple}Formula:}
                $$K_{eq} = K_1 + K_2 + \cdots + K_n$$

                \colorbox{orange!15}{\parbox{0.93\linewidth}{\footnotesize
                    Se tutte uguali: $K_{eq} = n \cdot K$
                }}

                \vspace{0.1cm}
                \colorbox{cyan!10}{\parbox{0.95\linewidth}{\tiny
                    \textbf{Es:} 3 nMOS con $K = 0.5$ mA/V² → $K_{eq} = 1.5$ mA/V²
                }}

                \vspace{0.15cm}
                \hrule
                \vspace{0.15cm}

                \textbf{\underline{2. MOSFET in SERIE}}

                \vspace{0.1cm}
                \colorbox{yellow!30}{\parbox{0.95\linewidth}{\footnotesize
                    • GATE in comune

                    • SOURCE equivalente = SOURCE più BASSO
                }}

                \vspace{0.1cm}
                \textbf{\color{purple}Formula:}
                $$\frac{1}{K_{eq}} = \frac{1}{K_1} + \frac{1}{K_2} + \cdots + \frac{1}{K_n}$$

                \colorbox{orange!15}{\parbox{0.93\linewidth}{\footnotesize
                    Per 2 MOS: $K_{eq} = \frac{K_1 \cdot K_2}{K_1 + K_2}$

                    \vspace{0.03cm}
                    Se uguali: $K_{eq} = \frac{K}{n}$
                }}

                \vspace{0.1cm}
                \colorbox{cyan!10}{\parbox{0.95\linewidth}{\tiny
                    \textbf{Es:} 2 nMOS $K_1=1$, $K_2=2$ mA/V² → $K_{eq}=0.67$ mA/V²
                }}

                \vspace{0.1cm}
                \colorbox{green!15}{\parbox{0.95\linewidth}{\tiny
                    \textbf{Fondamentale:} Queste semplificazioni evitano calcoli complessi nei circuiti.
                }}

            \end{minipage}
        };
        %------------ Semplificazioni MOSFET Header ---------------------
        \node[fancytitle, right=10pt] at (box.north west) {\color{white}Semplificazioni MOSFET};
    \end{tikzpicture}


    %------------ Analisi Porte Logiche ---------------
    \begin{tikzpicture}
        \node [mybox] (box){%
            \begin{minipage}{0.3\textwidth}

                \small

                \vspace{0.15cm}
                \colorbox{yellow!30}{\parbox{0.95\linewidth}{\footnotesize
                    \textbf{Quando usare:} Dopo aver fatto semplificazioni (serie/parallelo), quando $V_{DS} = V_{OUT}$ e devi capire la zona di funzionamento.
                }}

                \vspace{0.15cm}
                \colorbox{red!20}{\parbox{0.95\linewidth}{\footnotesize
                    \textbf{\color{red}$\bigstar$ IPOTESI PORTA LOGICA IDEALE}

                    Se ti hanno chiesto l'espressione logica della porta, puoi ipotizzare che sia \textbf{ideale}:

                    \vspace{0.05cm}
                    • $V_{OUT}$ ha valori logici \textbf{ALTO} e \textbf{BASSO}

                    • $V_{OUT} = V_{DS}$ del MOSFET (dopo semplificazioni)
                }}

                \vspace{0.15cm}
                \textbf{\underline{METODO:}}

                \vspace{0.1cm}
                \textbf{1. Uscita logica BASSA (``0'')}

                \vspace{0.05cm}
                \colorbox{cyan!10}{\parbox{0.93\linewidth}{\footnotesize
                    $V_{OUT} \approx 0V$ → $V_{DS}$ piccola

                    \vspace{0.03cm}
                    $\Rightarrow V_{DS} < V_{OV}$ → \textbf{ZONA OMICA}
                }}

                \vspace{0.15cm}
                \textbf{2. Uscita logica ALTA (``1'')}

                \vspace{0.05cm}
                \colorbox{orange!10}{\parbox{0.93\linewidth}{\footnotesize
                    $V_{OUT} \approx V_{DD}$ → $V_{DS}$ grande

                    \vspace{0.03cm}
                    $\Rightarrow V_{DS} > V_{OV}$ → \textbf{ZONA SATURAZIONE}
                }}

                \vspace{0.15cm}
                \colorbox{green!15}{\parbox{0.95\linewidth}{\tiny
                    \textbf{Nota:} Questo metodo ti permette di \textbf{ipotizzare} la zona di funzionamento senza fare calcoli complessi. Poi puoi verificare con le formule.
                }}

                \vspace{0.1cm}
                \colorbox{purple!10}{\parbox{0.95\linewidth}{\tiny
                    \textbf{Esempio pratico:}

                    Se $V_{OUT} = 0V$ (logica bassa) e hai $V_{OV} = 2V$:

                    $V_{DS} \approx 0V < 2V$ → OMICA $\checkmark$

                    \vspace{0.03cm}
                    Se $V_{OUT} = 5V$ (logica alta) e hai $V_{OV} = 2V$:

                    $V_{DS} \approx 5V > 2V$ → SATURAZIONE $\checkmark$
                }}

            \end{minipage}
        };
        %------------ Analisi Porte Logiche Header ---------------------
        \node[fancytitle, right=10pt] at (box.north west) {\color{white}Analisi Porte Logiche};
    \end{tikzpicture}


    %------------ Resistenza di canale ---------------
    \begin{tikzpicture}
        \node [mybox] (box){%
            \begin{minipage}{0.3\textwidth}

                \small
                \textbf{\color{blue}Resistenza di Canale ($R_{CH}$ o $R_{eq}$)}

                \vspace{0.15cm}
                \colorbox{yellow!30}{\parbox{0.95\linewidth}{\footnotesize
                    \textbf{Quando usare:} Calcolare la corrente nel MOSFET quando:

                    • $V_{OUT} = V_{DS}$ (l'uscita coincide con la tensione drain-source)

                    • $V_{OUT} \approx 0V$ (uscita logica bassa)
                }}

                \vspace{0.15cm}
                \colorbox{red!20}{\parbox{0.95\linewidth}{\footnotesize
                    \textbf{\color{red}$\bigstar$ DEFINIZIONE}

                    La \textbf{resistenza di canale} è la resistenza equivalente del MOSFET in un \textbf{intorno di $V_{DS} = 0V$}
                }}

                \vspace{0.15cm}
                \textbf{\underline{FORMULA:}}

                \vspace{0.1cm}
                \colorbox{cyan!20}{\parbox{0.95\linewidth}{
                    $$R_{CH} = R_{eq} = \frac{1}{2K \cdot V_{OV}}$$

                    \vspace{0.05cm}
                    \centering
                    dove $V_{OV} = V_{GS} - V_T$
                }}

                \vspace{0.15cm}
                \colorbox{yellow!20}{\parbox{0.95\linewidth}{\footnotesize
                    \textbf{Importante:} $K$ può essere:

                    • Il $K$ del singolo MOSFET

                    • Il $K_{eq}$ del MOSFET equivalente (dopo semplificazioni serie/parallelo)
                }}

                \vspace{0.15cm}
                \colorbox{orange!15}{\parbox{0.95\linewidth}{\footnotesize
                    \textbf{Origine:} Derivata di $I_D$ rispetto a $V_{DS}$ calcolata in $V_{DS}=0$ (approssimazione di Taylor al primo ordine)
                }}

                \vspace{0.15cm}
                \textbf{\underline{QUANDO È VALIDA:}}

                \vspace{0.05cm}
                \colorbox{green!15}{\parbox{0.95\linewidth}{\footnotesize
                    $\checkmark$ $V_{DS} \approx 0V$ (uscita logica bassa)

                    $\checkmark$ MOSFET in zona OMICA

                    $\checkmark$ Calcoli approssimativi di corrente
                }}

                \vspace{0.1cm}
                \colorbox{red!15}{\parbox{0.95\linewidth}{\footnotesize
                    $\times$ Se $V_{DS}$ NON è vicino a 0V

                    $\times$ In altri punti di lavoro (devi ricalcolare la derivata nel punto specifico)
                }}

                \vspace{0.15cm}
                \colorbox{red!20}{\parbox{0.95\linewidth}{\footnotesize
                    \textbf{\color{red}$\bigstar$ SANITY CHECK}

                    Dopo aver calcolato $I_D$ usando $R_{CH}$, \textbf{DEVI verificare}:

                    $$V_{R_{CH}} \ll V_{OV}$$

                    Dove $V_{R_{CH}}$ è la tensione ai capi della resistenza equivalente (= $V_{DS}$ del MOSFET).

                    \vspace{0.05cm}
                    Se $V_{R_{CH}} \approx V_{OV}$ o maggiore, l'approssimazione \textbf{NON è valida}!
                }}

                \vspace{0.15cm}
                \colorbox{purple!10}{\parbox{0.95\linewidth}{\tiny
                    \textbf{Esempio pratico:}

                    Se $K = 1$ mA/V², $V_{GS} = 3V$, $V_T = 1V$:

                    $V_{OV} = 3V - 1V = 2V$

                    $R_{CH} = \frac{1}{2 \cdot 1 \cdot 2} = \frac{1}{4}$ k$\Omega$ = 250 $\Omega$

                    \vspace{0.05cm}
                    Con $V_{DS} = 0.1V$:

                    $I_D \approx \frac{V_{DS}}{R_{CH}} = \frac{0.1V}{250\Omega} = 0.4$ mA

                    \vspace{0.05cm}
                    \textbf{Verifica:} $V_{DS} = 0.1V \ll V_{OV} = 2V$ $\checkmark$ OK!
                }}

            \end{minipage}
        };
        %------------ Resistenza di canale Header ---------------------
        \node[fancytitle, right=10pt] at (box.north west) {\color{white}Resistenza di canale};
    \end{tikzpicture}


    %------------ Carica condensatore con MOSFET ---------------
    \begin{tikzpicture}
        \node [mybox] (box){%
            \begin{minipage}{0.3\textwidth}

                \small

                \vspace{0.15cm}
                \colorbox{yellow!30}{\parbox{0.95\linewidth}{\footnotesize
                    \textbf{Scenario:} MOSFET utilizzato per caricare un condensatore (es. in porte logiche, circuiti di trasferimento carica)
                }}

                \vspace{0.15cm}
                \colorbox{red!20}{\parbox{0.95\linewidth}{\footnotesize
                    \textbf{\color{red}$\bigstar$ DIFFERENZA CRITICA pMOS vs nMOS}

                    La tensione massima raggiungibile sul condensatore dipende dal \textbf{tipo di MOSFET}!
                }}

                \vspace{0.15cm}
                \textbf{\underline{1. Carica con pMOS}}

                \vspace{0.1cm}
                \colorbox{green!20}{\parbox{0.95\linewidth}{\footnotesize
                    \textbf{Carica COMPLETA:} Il condensatore si carica fino a \textbf{$V_{DD}$}

                    \vspace{0.05cm}
                    $$V_{C,max} = V_{DD}$$
                }}

                \vspace{0.05cm}
                \colorbox{cyan!10}{\parbox{0.95\linewidth}{\tiny
                    \textbf{Motivo:} Nel pMOS, la corrente scorre da Source (alto) → Drain (basso).

                    \vspace{0.03cm}
                    Il pMOS può rimanere acceso fino a quando il condensatore raggiunge $V_{DD}$, perché il Source è collegato a $V_{DD}$ e mantiene sempre $V_{SG} > |V_T|$.
                }}

                \vspace{0.2cm}
                \hrule
                \vspace{0.15cm}

                \textbf{\underline{2. Carica con nMOS}}

                \vspace{0.1cm}
                \colorbox{orange!25}{\parbox{0.95\linewidth}{\footnotesize
                    \textbf{Carica LIMITATA:} Il condensatore si carica \textbf{solo fino a}:

                    \vspace{0.05cm}
                    $$V_{C,max} = V_G - V_T$$
                }}

                \vspace{0.05cm}
                \colorbox{cyan!10}{\parbox{0.95\linewidth}{\tiny
                    \textbf{Motivo:} Nel nMOS, quando il condensatore (collegato al Drain) si carica, aumenta $V_D$.

                    \vspace{0.03cm}
                    Quando $V_D$ raggiunge $V_G - V_T$, si ha $V_{GS} = V_{G} - V_{S} = V_{G} - (V_G - V_T) = V_T$ → il MOSFET \textbf{si spegne} (entra in interdizione).

                    \vspace{0.03cm}
                    \textbf{Non può caricare oltre} perché $V_{GS} = V_T$ è la condizione di soglia (OFF).
                }}

                \vspace{0.15cm}
                \colorbox{purple!15}{\parbox{0.95\linewidth}{\tiny
                    \textbf{Esempio pratico (CARICA):}

                    Se $V_G = 5V$ e $V_T = 1V$ per un nMOS:

                    $V_{C,max} = 5V - 1V = 4V$ (non $5V$!)

                    \vspace{0.05cm}
                    Con pMOS invece: $V_{C,max} = V_{DD}$ (carica completa)
                }}

                \vspace{0.25cm}
                \hrule
                \vspace{0.2cm}

                \textbf{\underline{SCARICA del condensatore (comportamento SPECULARE)}}

                \vspace{0.15cm}
                \textbf{\underline{1. Scarica con nMOS}}

                \vspace{0.1cm}
                \colorbox{green!20}{\parbox{0.95\linewidth}{\footnotesize
                    \textbf{Scarica COMPLETA:} Il condensatore si scarica fino a \textbf{GND (0V)}

                    \vspace{0.05cm}
                    $$V_{C,min} = 0V$$
                }}

                \vspace{0.05cm}
                \colorbox{cyan!10}{\parbox{0.95\linewidth}{\tiny
                    \textbf{Motivo:} Nel nMOS, il Source è collegato a GND e la corrente scorre dal condensatore (Drain) verso GND.

                    \vspace{0.03cm}
                    Il nMOS rimane acceso finché $V_{GS} > V_T$. Dato che $V_S = 0V$ (GND), finché $V_G > V_T$ il transistor resta acceso e può scaricare completamente il condensatore.
                }}

                \vspace{0.2cm}
                \textbf{\underline{2. Scarica con pMOS}}

                \vspace{0.1cm}
                \colorbox{orange!25}{\parbox{0.95\linewidth}{\footnotesize
                    \textbf{Scarica LIMITATA:} Il condensatore si scarica \textbf{solo fino a}:

                    \vspace{0.05cm}
                    $$V_{C,min} = V_G + |V_T|$$
                }}

                \vspace{0.05cm}
                \colorbox{cyan!10}{\parbox{0.95\linewidth}{\tiny
                    \textbf{Motivo:} Nel pMOS, quando il condensatore (collegato al Source) si scarica, diminuisce $V_S$.

                    \vspace{0.03cm}
                    Quando $V_S$ scende fino a $V_G + |V_T|$, si ha $V_{SG} = |V_T|$ → il MOSFET \textbf{si spegne}.

                    \vspace{0.03cm}
                    \textbf{Non può scaricare oltre} perché $V_{SG} = |V_T|$ è la condizione di soglia (OFF).
                }}

                \vspace{0.15cm}
                \colorbox{purple!15}{\parbox{0.95\linewidth}{\tiny
                    \textbf{Esempio pratico (SCARICA):}

                    Se $V_G = 2V$ e $|V_T| = 1V$ per un pMOS:

                    $V_{C,min} = 2V + 1V = 3V$ (non può scendere sotto!)

                    \vspace{0.05cm}
                    Con nMOS invece: $V_{C,min} = 0V$ (scarica completa)
                }}

                \vspace{0.2cm}
                \hrule
                \vspace{0.15cm}

                \colorbox{red!15}{\parbox{0.95\linewidth}{\footnotesize
                    \textbf{\color{red}$\triangle$ CONSEGUENZA PRATICA - Simmetria CARICA/SCARICA}

                    \vspace{0.05cm}
                    \textbf{CARICA:} pMOS completa ($\to V_{DD}$), nMOS limitata ($\to V_G - V_T$)

                    \textbf{SCARICA:} nMOS completa ($\to$ GND), pMOS limitata ($\to V_G + |V_T|$)

                    \vspace{0.05cm}
                    Nelle porte logiche cMOS:

                    • \textbf{pMOS} nella rete \textbf{pull-up} (PUN) → porta uscita a $V_{DD}$

                    • \textbf{nMOS} nella rete \textbf{pull-down} (PDN) → porta uscita a GND
                }}

            \end{minipage}
        };
        %------------ Carica/Scarica condensatore Header ---------------------
        \node[fancytitle, right=10pt] at (box.north west) {\color{white}Carica e Scarica di un condensatore con MOSFET};
    \end{tikzpicture}

    %%%%%%%%%%%%%%%%%%%%%%%%%%%%%%%%%%%%%%%%%%%%%%%%%%%%%%%%%%%%%%%%%%%%
    \columnbreak





    %------------ Tempo di propagazione ---------------
    \begin{tikzpicture}
        \node [mybox] (box){%
            \begin{minipage}{0.3\textwidth}

                \small
                \textbf{\color{blue}Tempo di propagazione ($\tau$ o $t_{prop}$)}

                \vspace{0.2cm}
                \textbf{Definizione:} Tempo impiegato a raggiungere la soglia della porta logica successiva.

                \vspace{0.15cm}
                \colorbox{yellow!20}{\parbox{0.95\linewidth}{
                    \textbf{Convenzione:} Se non specificato, si prende:
                    $$V_{finale} = \frac{V_{DD}}{2}$$
                }}

                \vspace{0.2cm}
                \textbf{\color{purple}Metodo 1: Approssimazione a corrente costante}

                \vspace{0.1cm}
                $$\tau = \frac{\Delta V \cdot C}{I_{sat}}$$

                \vspace{0.15cm}
                Dove:
                \begin{itemize}[itemsep=2pt, leftmargin=*]
                    \item $\Delta V = V_{finale} - V_{iniziale}$
                    \item $V_{finale} = \frac{V_{DD}}{2}$ (sempre!)
                    \item $C$ = capacità di carico
                    \item $I_{sat}$ = corrente di saturazione del MOSFET
                \end{itemize}

                \vspace{0.15cm}
                \colorbox{cyan!10}{\parbox{0.95\linewidth}{\footnotesize
                    \textbf{Esempio:} Se $V_{DD} = 5V$ e $V_{iniziale} = 0V$:

                    La transizione è da $0V$ a $\frac{5V}{2} = 2.5V$ (NON a 5V!)

                    $$\Delta V = 2.5V - 0V = 2.5V$$
                }}

            \end{minipage}
        };
        %------------ Tempo propagazione Header ---------------------
        \node[fancytitle, right=10pt] at (box.north west) {\color{white}Tempo di propagazione};
    \end{tikzpicture}

    %%%%%%%%%%%%%%%%%%%%%%%%%%%%%%%%%%%%%%%%%%%%%%%%%%%%%%%%%%%%%%%%%%%%

    %------------ Potenza statica ---------------
    \begin{tikzpicture}
        \node [mybox] (box){%
            \begin{minipage}{0.3\textwidth}

                \small
                \textbf{\color{blue}Potenza statica}

                \vspace{0.2cm}
                \textbf{Definizione:} Potenza consumata dal circuito quando gli ingressi e le uscite \textbf{NON commutano} (analisi statica).

                \vspace{0.2cm}
                \colorbox{yellow!20}{\parbox{0.95\linewidth}{
                    \textbf{Importante:} In analisi statica, il condensatore si comporta come se non ci fosse (circuito aperto).
                }}

                \vspace{0.2cm}
                \textbf{\color{purple}Formula:}

                \vspace{0.1cm}
                $$P_{statica} = I \cdot V_{DD}$$

                \vspace{0.15cm}
                Dove:
                \begin{itemize}[itemsep=2pt, leftmargin=*]
                    \item $I$ = corrente che scorre nel MOSFET/circuito
                    \item $V_{DD}$ = tensione di alimentazione
                \end{itemize}

                \vspace{0.15cm}
                \colorbox{cyan!10}{\parbox{0.95\linewidth}{\footnotesize
                    \textbf{Nota:} Poiché il condensatore è un circuito aperto in regime stazionario (nessun $\frac{dV}{dt}$), si calcola solo la corrente continua che scorre attraverso i MOSFET.
                }}

                \vspace{0.2cm}
                \colorbox{green!15}{\parbox{0.95\linewidth}{\footnotesize
                    \textbf{cMOS standard:} $P_{statica} = 0$ sempre.

                    Non esistono configurazioni che consumano potenza statica.
                }}

                \vspace{0.1cm}
                \colorbox{orange!15}{\parbox{0.95\linewidth}{\footnotesize
                    \textbf{cMOS non standard:} Possono avere configurazioni in cui $P_{statica} \neq 0$.
                }}

                \vspace{0.2cm}
                \colorbox{red!15}{\parbox{0.95\linewidth}{\footnotesize
                    \textbf{\color{red}$\bigstar$ IMPORTANTE - Calcolo $V_{GS}$}

                    In analisi statica, se il \textbf{source dell'nMOS NON è a massa} (ma collegato a un'altra alimentazione):

                    \textbf{NON} usare $V_G$ direttamente, ma calcolare:
                    $$V_{GS} = V_G - V_S$$

                    Lo stesso vale per pMOS se il source NON è a $V_{DD}$.
                }}

            \end{minipage}
        };
        %------------ Potenza statica Header ---------------------
        \node[fancytitle, right=10pt] at (box.north west) {\color{white}Potenza statica};
    \end{tikzpicture}

    %%%%%%%%%%%%%%%%%%%%%%%%%%%%%%%%%%%%%%%%%%%%%%%%%%%%%%%%%%%%%%%%%%%%

    %------------ Potenza dinamica ---------------
    \begin{tikzpicture}
        \node [mybox] (box){%
            \begin{minipage}{0.3\textwidth}

                \small
                \textbf{\color{blue}Potenza dinamica}

                \vspace{0.2cm}
                \textbf{Definizione:} Potenza consumata durante le commutazioni degli ingressi uscite.

                \vspace{0.2cm}
                \colorbox{red!20}{\parbox{0.95\linewidth}{\footnotesize
                    \textbf{\color{red}$\bigstar$ CONDIZIONE FONDAMENTALE}

                    \vspace{0.1cm}
                    Prima di applicare la formula, verificare che:
                    $$\tau_{prop} \leq \frac{T_{in}}{2}$$

                    Dove:
                    \begin{itemize}[itemsep=1pt]
                        \item $\tau_{prop}$ = tempo di propagazione
                        \item $T_{in}$ = periodo del segnale di ingresso
                    \end{itemize}

                    \vspace{0.05cm}
                    Se $\tau_{prop} > \frac{T_{in}}{2}$, il circuito \textbf{NON ha tempo} di raggiungere il regime prima della prossima commutazione $\Rightarrow$ la formula \textbf{NON è valida}.

                    \vspace{0.15cm}
                    \colorbox{cyan!10}{\parbox{0.92\linewidth}{\tiny
                        \textbf{Nota pratica:} Se hai calcolato $\tau_{prop}$ per una transizione (es. high$\to$low) ma la potenza dinamica riguarda la transizione opposta (low$\to$high), verifica l'\textbf{ordine di grandezza}.

                        \vspace{0.05cm}
                        \textbf{Se $K_n$ e $K_p$ sono comparabili numericamente:} I due tempi di propagazione saranno multipli ma \textbf{stesso ordine di grandezza}. Se $\tau_{prop} \ll \frac{T_{in}}{2}$ (molto minore), sei a posto anche senza calcolare l'altro!

                        \vspace{0.05cm}
                        \textbf{\color{red}ATTENZIONE:} Questa assunzione vale \textbf{SOLO se $K_n \approx K_p$}. Se i valori di $K$ sono molto diversi, devi calcolare entrambi i tempi di propagazione.
                    }}
                }}

                \vspace{0.2cm}
                \textbf{\color{purple}Formula generale:}

                \vspace{0.1cm}
                $$P_D = V_{DD} \sum_{i} (V_{OH,i} - V_{OL,i}) \cdot C_i \cdot f_i$$

                \vspace{0.2cm}
                \colorbox{yellow!20}{\parbox{0.95\linewidth}{
                    \textbf{Caso semplificato} (un solo nodo d'uscita):
                    $$P_D = V_{DD} \cdot (V_{OH} - V_{OL}) \cdot C_L \cdot f_{out}$$
                }}

                \vspace{0.2cm}
                Dove:
                \begin{itemize}[itemsep=2pt, leftmargin=*]
                    \item $V_{DD}$ = tensione di alimentazione
                    \item $V_{OH}$ = tensione output HIGH (valore massimo)
                    \item $V_{OL}$ = tensione output LOW (valore minimo)
                    \item $C_L$ = capacità del carico
                    \item $f_{out}$ = frequenza di uscita
                \end{itemize}

                \vspace{0.2cm}
                \colorbox{cyan!10}{\parbox{0.95\linewidth}{\footnotesize
                    \textbf{Come determinare $V_{OH}$ e $V_{OL}$:}

                    Sono i valori massimo e minimo dell'uscita durante le commutazioni.

                    \textbf{Metodi:}
                    \begin{itemize}[itemsep=1pt]
                        \item Dal grafico di $V_{out}(t)$ (se richiesto in precedenza)
                        \item Forniti direttamente nel testo dell'esercizio
                        \item Analizzando le transizioni del circuito
                    \end{itemize}
                }}

            \end{minipage}
        };
        %------------ Potenza dinamica Header ---------------------
        \node[fancytitle, right=10pt] at (box.north west) {\color{white}Potenza dinamica};
    \end{tikzpicture}

    \newpage
    %%%%%%%%%%%%%%%%%%%%%%%%%%%%%%%%%%%%%%%%%%%%%%%%%%%%%%%%%%%%%%%%%%%%

    %------------ cMOS ---------------
    \label{cMOS}
    \begin{tikzpicture}
        \node [mybox] (box){%
            \begin{minipage}{0.3\textwidth}

                \small

                \vspace{0.15cm}
                \colorbox{yellow!30}{\parbox{0.95\linewidth}{\footnotesize
                    \textbf{Definizione:} Una porta logica \textbf{cMOS} (Complementary MOS) è composta da due reti complementari:

                    \vspace{0.05cm}
                    • \textbf{PUN} (Pull-Up Network): rete di \textbf{pMOS}

                    • \textbf{PDN} (Pull-Down Network): rete di \textbf{nMOS}
                }}

                \vspace{0.15cm}
                \colorbox{red!20}{\parbox{0.95\linewidth}{\footnotesize
                    \textbf{\color{red}$\bigstar$ REGOLA FONDAMENTALE}

                    \vspace{0.05cm}
                    In qualsiasi configurazione di ingresso:

                    \vspace{0.03cm}
                    \textbf{Solo UNA rete è attiva (ON) alla volta}

                    \vspace{0.05cm}
                    • Se PUN è ON → PDN è OFF (uscita = $V_{DD}$)

                    • Se PDN è ON → PUN è OFF (uscita = GND)
                }}

                \vspace{0.15cm}
                \textbf{\underline{Significato PRATICO negli esercizi:}}

                \vspace{0.05cm}
                \colorbox{green!15}{\parbox{0.95\linewidth}{\footnotesize
                    \textbf{1. Potenza statica = 0}

                    Poiché una rete è sempre OFF, non c'è percorso diretto tra $V_{DD}$ e GND → $P_{statica} = 0$
                }}

                \vspace{0.1cm}
                \colorbox{cyan!10}{\parbox{0.95\linewidth}{\footnotesize
                    \textbf{2. Analisi per stati logici}

                    Per ogni combinazione di ingressi, verifica:

                    • Quali MOSFET sono ON/OFF

                    • Quale rete (PUN o PDN) è attiva

                    • Output = $V_{DD}$ se PUN ON, = GND se PDN ON
                }}

                \vspace{0.15cm}
                \textbf{\underline{Esempio: cMOS Inverter}}

                \vspace{0.05cm}
                \colorbox{purple!10}{\parbox{0.95\linewidth}{\tiny
                    \textbf{Ingresso ALTO (``1''):}

                    • nMOS ON → PDN attiva → Uscita = GND (``0'')

                    • pMOS OFF → PUN spenta

                    \vspace{0.05cm}
                    \textbf{Ingresso BASSO (``0''):}

                    • pMOS ON → PUN attiva → Uscita = $V_{DD}$ (``1'')

                    • nMOS OFF → PDN spenta
                }}

                \vspace{0.1cm}
                \colorbox{orange!15}{\parbox{0.95\linewidth}{\tiny
                    \textbf{Nota:} Le reti sono \textbf{complementari}: se PUN realizza $f$, PDN realizza $\overline{f}$
                }}

            \end{minipage}
        };
        %------------ cMOS Header ---------------------
        \node[fancytitle, right=10pt] at (box.north west) {\color{white}Porte cMOS - Definizione};
    \end{tikzpicture}


    %------------ Costruzione PUN da PDN ---------------
    \begin{tikzpicture}
        \node [mybox] (box){%
            \begin{minipage}{0.3\textwidth}

                \small

                \vspace{0.15cm}
                \colorbox{yellow!30}{\parbox{0.95\linewidth}{\footnotesize
                    \textbf{Problema:} Data la rete Pull-Down (PDN) con nMOS, costruire la rete Pull-Up (PUN) con pMOS
                }}

                \vspace{0.15cm}
                \colorbox{red!20}{\parbox{0.95\linewidth}{\footnotesize
                    \textbf{\color{red}$\bigstar$ METODO - Trasformazione DUALE}

                    \vspace{0.05cm}
                    \colorbox{yellow!40}{\parbox{0.92\linewidth}{\tiny
                        \centering
                        \textbf{In pratica: INVERSIONE RICORSIVA di SERIE e PARALLELO}
                    }}

                    \vspace{0.05cm}
                    Dalla PDN alla PUN:

                    \vspace{0.05cm}
                    \textbf{1.} SERIE → PARALLELO

                    \textbf{2.} PARALLELO → SERIE

                    \textbf{3.} nMOS → pMOS

                    \textbf{4.} Gate (ingressi) → RIMANGONO UGUALI
                }}

                \vspace{0.15cm}
                \textbf{\underline{PROCEDURA MECCANICA:}}

                \vspace{0.05cm}
                \colorbox{cyan!10}{\parbox{0.95\linewidth}{\footnotesize
                    \textbf{Step 1:} Identifica la struttura della PDN

                    • Individua le connessioni SERIE

                    • Individua le connessioni PARALLELO

                    \vspace{0.05cm}
                    \textbf{Step 2:} Applica la trasformazione

                    • Ogni SERIE diventa PARALLELO

                    • Ogni PARALLELO diventa SERIE

                    • Sostituisci nMOS con pMOS

                    • Mantieni gli stessi gate
                }}

                \vspace{0.15cm}
                \colorbox{purple!10}{\parbox{0.95\linewidth}{\tiny
                    \textbf{Esempio pratico:}

                    \textbf{PDN:} nMOS(A) in SERIE con [nMOS(B) || nMOS(C)]

                    \vspace{0.05cm}
                    \textbf{Applicazione trasformazione:}

                    • A in SERIE → A in PARALLELO

                    • (B || C) → (B in SERIE con C)

                    \vspace{0.05cm}
                    \textbf{PUN:} pMOS(A) in PARALLELO con [pMOS(B) in SERIE con pMOS(C)]

                    \vspace{0.05cm}
                    In formula: $PUN = A \parallel (B \cdot C)$
                }}

                \vspace{0.15cm}
                \colorbox{green!15}{\parbox{0.95\linewidth}{\footnotesize
                    \textbf{Verifica:} Le due reti sono complementari

                    • PDN: $f = A \cdot (B + C)$

                    • PUN: $\overline{f} = \overline{A} + (\overline{B} \cdot \overline{C}) = \overline{A \cdot (B + C)}$ $\checkmark$
                }}

                \vspace{0.1cm}
                \colorbox{orange!15}{\parbox{0.95\linewidth}{\tiny
                    \textbf{Nota:} Questo metodo garantisce che solo una rete sia ON alla volta (proprietà fondamentale delle porte cMOS)
                }}

            \end{minipage}
        };
        %------------ Costruzione PUN da PDN Header ---------------------
        \node[fancytitle, right=10pt] at (box.north west) {\color{white}Costruzione PUN da PDN};
    \end{tikzpicture}
    %%%%%%%%%%%%%%%%%%%%%%%%%%%%%%%%%%%%%%%%%%%%%%%%%%%%%%%%%%%%%%%%%%%%

    \newpage

    %------------ AND/nAND ---------------
    \label{AND/nAND}
    \begin{tikzpicture}
        \node [mybox] (box){%
            \begin{minipage}{0.3\textwidth}
                \noindent \textbf{AND (AND/nAND):}
                \begin{enumerate}
                    \item \textbf{Entrata:}
                          \begin{equation*}
                              \begin{aligned}
                                   & A = 0, \quad B = 0 \\
                                   & \text{Uscita: } 0
                              \end{aligned}
                          \end{equation*}

                    \item
                          \begin{equation*}
                              \begin{aligned}
                                   & A = 0, \quad B = 1 \\
                                   & \text{Uscita: } 0
                              \end{aligned}
                          \end{equation*}

                    \item
                          \begin{equation*}
                              \begin{aligned}
                                   & A = 1, \quad B = 0 \\
                                   & \text{Uscita: } 0
                              \end{aligned}
                          \end{equation*}

                    \item
                          \begin{equation*}
                              \begin{aligned}
                                   & A = 1, \quad B = 1 \\
                                   & \text{Uscita: } 1
                              \end{aligned}
                          \end{equation*}
                \end{enumerate}

            \end{minipage}
        };
        %------------ AND/nAND Header ---------------------
        \node[fancytitle, right=10pt] at (box.north west) {\color{white}AND/nAND};
    \end{tikzpicture}

    %------------ OR/nOR ---------------
    \label{OR/nOR}
    \begin{tikzpicture}
        \node [mybox] (box){%
            \begin{minipage}{0.3\textwidth}
                \noindent \textbf{OR (nOR):}
                \begin{enumerate}
                    \item \textbf{Entrata:}
                          \begin{equation*}
                              \begin{aligned}
                                   & A = 0, \quad B = 0 \\
                                   & \text{Uscita: } 0
                              \end{aligned}
                          \end{equation*}

                    \item
                          \begin{equation*}
                              \begin{aligned}
                                   & A = 0, \quad B = 1 \\
                                   & \text{Uscita: } 1
                              \end{aligned}
                          \end{equation*}

                    \item
                          \begin{equation*}
                              \begin{aligned}
                                   & A = 1, \quad B = 0 \\
                                   & \text{Uscita: } 1
                              \end{aligned}
                          \end{equation*}

                    \item
                          \begin{equation*}
                              \begin{aligned}
                                   & A = 1, \quad B = 1 \\
                                   & \text{Uscita: } 1
                              \end{aligned}
                          \end{equation*}
                \end{enumerate}

            \end{minipage}
        };
        %------------ OR/nOR Header ---------------------
        \node[fancytitle, right=10pt] at (box.north west) {\color{white}OR/nOR};
    \end{tikzpicture}

    %------------ XOR/nXOR ---------------
    \label{XOR/nXOR}
    \begin{tikzpicture}
        \node [mybox] (box){%
            \begin{minipage}{0.3\textwidth}
                \noindent \textbf{XOR (nXOR):}
                \begin{enumerate}
                    \item \textbf{Entrata:}
                          \begin{equation*}
                              \begin{aligned}
                                   & A = 0, \quad B = 0 \\
                                   & \text{Uscita: } 0
                              \end{aligned}
                          \end{equation*}

                    \item
                          \begin{equation*}
                              \begin{aligned}
                                   & A = 0, \quad B = 1 \\
                                   & \text{Uscita: } 1
                              \end{aligned}
                          \end{equation*}

                    \item
                          \begin{equation*}
                              \begin{aligned}
                                   & A = 1, \quad B = 0 \\
                                   & \text{Uscita: } 1
                              \end{aligned}
                          \end{equation*}

                    \item
                          \begin{equation*}
                              \begin{aligned}
                                   & A = 1, \quad B = 1 \\
                                   & \text{Uscita: } 0
                              \end{aligned}
                          \end{equation*}
                \end{enumerate}

            \end{minipage}
        };
        %------------ XOR/nXOR Header ---------------------
        \node[fancytitle, right=10pt] at (box.north west) {\color{white}XOR/nXOR};
    \end{tikzpicture}

    %------------ Box 17 ---------------
    \label{Box 17}
    \begin{tikzpicture}
        \node [mybox] (box){%
            \begin{minipage}{0.3\textwidth}
                \dummytext{Content for Box 17}
            \end{minipage}
        };
        %------------ Box 17 Header ---------------------
        \node[fancytitle, right=10pt] at (box.north west) {\color{white}Box 17};
    \end{tikzpicture}


    %------------ Box 18 ---------------
    \label{Box 18}
    \begin{tikzpicture}
        \node [mybox] (box){%
            \begin{minipage}{0.3\textwidth}
                \dummytext{Content for Box 18}
            \end{minipage}
        };
        %------------ Box 18 Header ---------------------
        \node[fancytitle, right=10pt] at (box.north west) {\color{white}Box 18};
    \end{tikzpicture}

    %------------ Box 19 ---------------
    \label{Box 19}
    \begin{tikzpicture}
        \node [mybox] (box){%
            \begin{minipage}{0.3\textwidth}
                \dummytext{Content for Box 19}
            \end{minipage}
        };
        %------------ Box 19 Header ---------------------
        \node[fancytitle, right=10pt] at (box.north west) {\color{white}Box 19};
    \end{tikzpicture}

    %------------ Box 20 ---------------
    \label{Box 20}
    \begin{tikzpicture}
        \node [mybox] (box){%
            \begin{minipage}{0.3\textwidth}
                \dummytext{Content for Box 20}
            \end{minipage}
        };
        %------------ Box 20 Header ---------------------
        \node[fancytitle, right=10pt] at (box.north west) {\color{white}Box 20};
    \end{tikzpicture}

    %------------ Box 21 ---------------
    \label{Box 21}
    \begin{tikzpicture}
        \node [mybox] (box){%
            \begin{minipage}{0.3\textwidth}
                \dummytext{Content for Box 21}
            \end{minipage}
        };
        %------------ Box 21 Header ---------------------
        \node[fancytitle, right=10pt] at (box.north west) {\color{white}Box 21};
    \end{tikzpicture}

    %------------ Box 22 ---------------
    \label{Box 22}
    \begin{tikzpicture}
        \node [mybox] (box){%
            \begin{minipage}{0.3\textwidth}
                \dummytext{Content for Box 22}
            \end{minipage}
        };
        %------------ Box 22 Header ---------------------
        \node[fancytitle, right=10pt] at (box.north west) {\color{white}Box 22};
    \end{tikzpicture}

    %------------ Box 23 ---------------
    \label{Box 23}
    \begin{tikzpicture}
        \node [mybox] (box){%
            \begin{minipage}{0.3\textwidth}
                \dummytext{Content for Box 23}
            \end{minipage}
        };
        %------------ Box 23 Header ---------------------
        \node[fancytitle, right=10pt] at (box.north west) {\color{white}Box 23};
    \end{tikzpicture}

\end{multicols*}
\end{document}